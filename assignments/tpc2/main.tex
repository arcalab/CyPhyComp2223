\documentclass[a4paper, 11pt]{article}

%% packages
\usepackage{fullpage} % changes the margin
\usepackage{hyperref} % Links
\usepackage[utf8]{inputenc}
\usepackage{lmodern}
\usepackage{amsfonts}
\usepackage{amsthm}
\usepackage{amsmath}
\usepackage{braket}
\usepackage{proof}
\usepackage{fullpage}
%%

%% macros
\newcommand{\complex}{\mathbb{C}}
\newcommand{\vecs}{\mathcal{V}}
\newcommand{\id}{\mathrm{id}}
%% environments
\theoremstyle{definition}
\newtheorem{definition}{Definition}
\newtheorem{examples}{Example}
\newtheorem{exercises}{Exercises}
\newtheorem{exercise}{Exercise}
\newtheorem{postulate}{Postulate}
\newtheoremstyle{sub}% name
{2pt}        % Space above, empty = `usual value'
{2pt}        % Space below
{}              % Body font
{}    % Indent amount (empty = no indent, \parindent = para indent)
{}              % Thm head font
{.}    % Punctuation after thm head
{0.5em}         % Space after thm head: \newline = linebreak
{{\thmname{#1}\thmnumber{ #2}}\thmnote{ (#3)}}  % Thm head spec
\theoremstyle{sub}
\newtheorem{subexercise}{Part}[exercise]
\usepackage[linewidth=1pt]{mdframed}
\newcommand{\conc}{\mathbin{+\mkern-08mu+}}

%% config
\date{}
\linespread{1.10}
%%

\begin{document}

\title{Cyber-Physical Programming\\ \large TPC-2}
\author{Renato Neves and José Proença \\ \scriptsize
  \href{mailto:nevrenato@di.uminho.pt}{nevrenato@di.uminho.pt}}
\maketitle

\newcommand{\blue}[1]{\textcolor{blue}{#1}}
It is often necessary to incorporate \emph{message logs} in whatever
programming language we are working with. For example, we might wish to register
the speed of a car periodically. So let us consider the following simple,
imperative programming language:
\[
        \mathtt{Prog}(X) \ni \mathtt{x\ \blue{:=}\ t} \mid
        \mathtt{\blue{write_m}(p)} \mid
	\mathtt{p \> \blue{;} \> q} \mid
	\mathtt{\blue{if} \> b \> \blue{then} \> p \> \blue{else} \> q} \mid
	\mathtt{\blue{while} \> b \> \blue{do} \> \{ \> p \> \}}
\]
Note that $\mathtt{t}$ is a linear term (defined in previous lectures) and
$\mathtt{m}$ in the construct $\blue{\mathtt{write_m}}$ is a list of messages.
The program $\mathtt{\blue{write_m}(p)}$ reads as ``write messages $\mathtt{m}$
and then run program $\mathtt{p}$".  For such a language we take a semantics
$\langle \mathtt{p}, \sigma \rangle \Downarrow m, \sigma'$ which informs not
only of the output of $\mathtt{p}$ (i.e. $\sigma'$) but also presents a list of
messages (i.e. $m$). Specifically, we adopt the following semantic rules:
\begin{flalign*}
        \infer[(\text{asg})]{
                \langle \mathtt{x\ \blue{:=}\ t}, \sigma \rangle \Downarrow 
                [\,], \sigma[\mathtt{r}/\mathtt{x}]
        }{
                \langle \mathtt{t}, \sigma \rangle \Downarrow \mathtt{r}
                }
        \hspace{2cm}
        \infer[(\text{wait})]{
                \langle \mathtt{\blue{wait_m}(p)}, \sigma \rangle
                \Downarrow m \conc n, \sigma'
        }{
                \langle \mathtt{p}, \sigma \rangle \Downarrow n, \sigma'      
                }
\end{flalign*}
\begin{flalign*}
        \infer[(\text{seq})]{
                \langle \mathtt{p\ \blue{;}\ q}, \sigma \rangle \Downarrow
                m \conc n, \sigma'' 
        }{
                \langle \mathtt{p}, \sigma \rangle \Downarrow m, \sigma' 
                \qquad
                \langle \mathtt{q}, \sigma' \rangle \Downarrow n, \sigma''
        }
\end{flalign*}
\begin{flalign*}
        \infer[(\text{if$_1$})]{
                \langle \mathtt{\blue{if} \> b \> \blue{then} \> \> p \> \blue{else} \> q}, 
                \sigma \rangle \Downarrow m, \sigma'
        }{
                \langle \mathtt{b}, \sigma \rangle \Downarrow \mathtt{tt} \qquad
                \langle \mathtt{p}, \sigma \rangle \Downarrow m, \sigma'
        } \hspace{1cm} 
        \infer[(\text{if$_2$})]{
                \langle \mathtt{\blue{if} \> b \> \blue{then} \> \> p \> \blue{else} \> q}, 
                \sigma \rangle \Downarrow m, \sigma'
        }{
                \langle \mathtt{b}, \sigma \rangle \Downarrow \mathtt{ff} \qquad
                \langle \mathtt{q}, \sigma \rangle \Downarrow m, \sigma'
        }
\end{flalign*}
\begin{flalign*}
        \infer[(\text{wh$_1$})]{
                \langle \mathtt{\blue{while} \> b \> \blue{do} \> \{ \> p \> \}}, \sigma \rangle
                \Downarrow m \conc n, \sigma''
        }{
                \langle \mathtt{b}, \sigma \rangle \Downarrow \mathtt{tt} \qquad
                \langle \mathtt{p}, \sigma \rangle \Downarrow m, \sigma' \qquad
                \langle \mathtt{\blue{while} \> b \> \blue{do} \> \{ \> p \> \}}, \sigma'
                \rangle \Downarrow n, \sigma'' 
        }
\end{flalign*}
\begin{flalign*}
        \infer[(\text{wh$_2$})]{
                \langle \mathtt{\blue{while} \> b \> \blue{do} \> \{ \> p \> \}}, \sigma \rangle
                \Downarrow [ \, ], \sigma
        }{
                \langle \mathtt{b}, \sigma \rangle \Downarrow \mathtt{ff}
        }
\end{flalign*}

We can then define a natural notion of \emph{equivalence} for our programs: 
we say that two programs \texttt{p} and \texttt{q} are equivalent 
(in symbols, $\mathtt{p} \sim \mathtt{q}$) if for all environments $\sigma$ we have
\[
        \langle \mathtt{p}, \sigma \rangle \Downarrow m, \sigma' \text{ iff }
        \langle \mathtt{q}, \sigma \rangle \Downarrow m, \sigma'
\]

\begin{exercise}
        Prove that $\mathtt{\blue{write_m}(\blue{write_n}(p))} \sim  
        \mathtt{\blue{write_{m \conc n}}(p)}$. Can you think of (and prove)
        other interesting equivalences? Note that the more equivalences a compiler
        knows the more ways it has to do program optimisations.
\end{exercise}

\begin{exercise}
        Implement in \texttt{Haskell} the while-language described above and
        its semantics. Suggestion: use the code developed in previous
        lectures.
\end{exercise}

\begin{mdframed}
  What to submit: A \texttt{.pdf} file containing the solution to
  the first exercise and a also \texttt{.hs} file containing the code that you developed
  (properly commented!) for the second exercise. Please send a corresponding
  \texttt{.zip} archive by email
  (\texttt{nevrenato@di.uminho.pt}) with the name
  ``\texttt{cpp2223-N.zip}'', where ``\texttt{N}'' is your student
  number.  The subject of the email should be ``\texttt{cpp2223 N
    TPC-2}''.
\end{mdframed}



%% Bibliography
\bibliographystyle{alpha}
\bibliography{biblioTeaching}

\end{document}
