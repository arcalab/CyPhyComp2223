\documentclass[11pt]{article}

\usepackage{graphicx,amsmath}
\usepackage{stmaryrd} % cf. interleave
\usepackage{booktabs}
\usepackage{amscd}
\usepackage{multicol}
\usepackage[absolute,overlay]{textpos}
\usepackage{alltt}
\usepackage{proof}
\usepackage{subcaption}
\usepackage{tikz}
\usepackage{tikz-cd}
\usepackage[new]{old-arrows}
\usepackage[all]{xy}
\usepackage{pgfplots}
\usepackage{textcomp}
\usepackage{amssymb} % lesssim

\input{macros}

\date{Cyber-Physical Computation  2022/2023\\ \large TPC-1}

\begin{document}
 
% --------------------------------------------------------------
%                         Start here
% --------------------------------------------------------------
 
\title{CCS and equivalences}
\author{Jos\'{e} Proen\c{c}a \& Renato Neves
\\
pro@isep.ipp.pt \& nevrenato@di.uminho.pt}
%\\
%Arquitectura e C\'alculo -- 2015/2016} 
 
\maketitle

\descrbox{To do}{
  Solve the exercises and produce a PDF with your answers. 
}

\descrbox{How to submit via email}{
Please send it by email
  (\texttt{pro@isep.ipp.pt}) with the name
  ``\texttt{cpc2223-N.pdf}'', where ``\texttt{N}'' is your student
  number.  The subject of the email should be ``\texttt{cpc2223 N
    TPC-1}''.
}

\descrbox{Valorization questions}{
  Questions marked with \underline{\textbf{[Hard]}} are valorization questions. These have very small marks when compared with the other questions and are meant to be more difficult.
}

\descrbox{Deadline}{
  7 April 2023 @ 23:59 (Friday) 
}



\newcommand{\compn}[1]{\textsf{\textcolor{purple}{#1}}\xspace}
\newcommand{\chn}[1]{\textsf{\textcolor{teal}{#1}}\xspace}

\section*{CCS analysis}



% \[\includegraphics[width=\textwidth]{RFID-conveyor.pdf}\]

\begin{myExercise} \label{ex:procs}
  For each of the CCS processes below, \textbf{draw} its transition system.
   % and \emph{explain} informally what it does.
  \subex{$A = a.b.0$}
  \subex{$B = A + a.0$}
  \subex{$C = (B || c.d.A)\backslash\{d\}$}
\end{myExercise}

\begin{myExercise}
  Recall $A$ and $B$ processes from \cref{ex:procs}.
  % Check if the following holds, and prove it:
  \subex{\textbf{Prove} that $A \lesssim B$ or \textbf{explain} why not.}
  \subex{\textbf{Prove} that $B \lesssim A$ or \textbf{explain} why not.}
  \subex{\textbf{Prove} that $A \sim B$ or \textbf{explain} why not.}
\end{myExercise}

\begin{myExercise}
  \underline{\textbf{[Hard]}} \textbf{Prove} that, for all CCS processes $P$ and $Q$: $$P+Q ~~\sim~~ Q+P$$ 
\end{myExercise}


\section*{CCS modelling}

% \begin{myExercise} \label{ex:model}
% Consider a system with 3 interacting processes -- \textbf{A}, \textbf{B}, and \textbf{V} -- such that:
% \begin{itemize}
%   \item A \textbf{V}ending machine repeatedly receives a \underline{coin}, and returns either a \underline{chocolate}, an \underline{apple}, or the \underline{coin} back.
%   \item \textbf{A}lice repeatedly pays and gets \underline{chocolate}s;
%   \item \textbf{B}ob repeatedly pays and gets \underline{apple}s and \underline{chocolate}s: after every 2 \underline{apple}s he gets a \underline{chocolate};
% \end{itemize}
%    \subex{ Specify this system using CCS.}
%    \subex{ Draw its transition system.}
% \end{myExercise}

\begin{myExercise} \label{ex:model2}
Consider the 5 components below.
\begin{itemize}
  \item \textbf{T}: A temperature sensor that periodically sends a \underline{temperature} value;
  \item \textbf{H}: A humidity sensor that periodically sends a \underline{humidity} value;
  \item \textbf{C}: A clock that sends a \underline{timestamp} with the current time;
  \item \textbf{O}: An orchestrator that receives a \underline{temperature} value, followed by a \underline{humidity} value and  by a \underline{timestamp}, and in the end sends this \underline{data} package;
  \item \textbf{D}: A display that receives \underline{data} from the orchestrator and \underline{display}s the content.
\end{itemize}
Consider each \underline{underlined} word above to be an action of our CSS processes.

\subex{\textbf{Specify} each of these 5 components in CCS and \textbf{draw} their transition system.}
\subex{\textbf{Specify} a new component \textbf{S} of this system, which composes the 5 components above in parallel, imposing synchronisation of all actions except \underline{display}.}
\subex{\textbf{Propose} a variation of a similar system \textbf{S2} in CCS with no orchestrator. In this variation:
\begin{enumerate}
\item the humidity sensor informs the temperature sensor, then
\item the temperature sensor informs the timestamp, then
\item the timestamp sends the whole data to the display; and finally
\item the display prompts the humidity sensor to restart the process.
\end{enumerate}
% I.e., that executes in parallel a variation of the components \textbf{T}, \textbf{H}, \textbf{C}, and \textbf{D}, which exchange messages am }
}
\subex{\underline{\textbf{[Hard]}} Experiment with the tool mCRL2 (\url{https://mcrl2.org}). Use it to \textbf{validate} your \textbf{S} and \textbf{S2} definitions above.}
\end{myExercise}


 
\end{document}