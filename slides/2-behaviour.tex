\documentclass[aspectratio=169]{beamer}
\usepackage{etex} % fixes new-dimension error
\usepackage{lmodern}
\usepackage[T1]{fontenc}

%----------------------------------------------------------------------------
\usepackage{graphicx,amsmath}
\usepackage{stmaryrd} % cf. interleave
\usepackage{booktabs}
\usepackage{amscd}
\usepackage{multicol}
\usepackage[absolute,overlay]{textpos}
\usepackage{alltt}
\usepackage{proof}
\usepackage{subcaption}
\usepackage{tikz}
\usepackage{tikz-cd}
\usepackage[new]{old-arrows}
\usepackage[all]{xy}
\usepackage{pgfplots}
\usepackage{textcomp}

\input{macros/macros}
%-------------- template --------------------------------------------------
\input{macros/beamerconf}

%------ using pstricks (rnode etc) ------------------------------------------
% \usepackage{pstricks,pst-node,pst-text,pst-3d}


% % context
% \AtBeginSection[]
% {
%     \begin{frame}
%         \frametitle{Table of Contents}
%         \tableofcontents[currentsection]
%     \end{frame}
% }
% \author[Renato Neves]{Renato Neves}
% % logos of institutions
% \titlegraphic{
%   \begin{textblock*}{5cm}(6.7cm,7.45cm)
%      \includegraphics[scale=0.06]{images/uminho.png}\hspace*{.85cm}~%
%   \end{textblock*}
%   \begin{textblock*}{5cm}(9.4cm,7.45cm)
%     \includegraphics[scale=0.50]{images/haslab.pdf}
%   \end{textblock*}
% }
% % No date
% \date{}

\begin{document}

% \setLectureBasic{Cyber-Physical Computation}
\setLecture{2}{Transition Systems}
\frame[plain]{\titlepage}


\section{What are Formal Methods?}


\begin{slide}{What are formal methods?}
  \centering

  \begin{minipage}{0.7\textwidth}
  \Large
  \begin{block}{}
    \centering
    Formal methods are \structure{techniques} to
    \\
    model \alert{complex systems} using
    \\
    \alert{rigorous mathematical models}

  \end{block}
  \end{minipage}
  \\[10mm]

  \begin{minipage}{0.32\textwidth}
  \begin{alertblock}{Specification}
    Define part of the system using a modelling language
  \end{alertblock}
  \end{minipage}
~~
  \begin{minipage}{0.3\textwidth}
  \begin{alertblock}{Verification}
    Prove properties.\\Show correctness.\\Find bugs.
  \end{alertblock}
  \end{minipage}
~~
  \begin{minipage}{0.28\textwidth}
  \begin{exampleblock}{Implementation}
    Generate correct code.\\~\\
  \end{exampleblock}
  \end{minipage}

\end{slide}

\begin{frame}
  \huge\centering
  All formal models are \alert{wrong}
  \pause
  \\[5mm]
  ... but some of them are \structure{usefull}!
\end{frame}


% \begin{slide}{Many success cases}
%   \huge\centering

%   Cryptographic protocols
%   \\[5mm]
%   Distributed systems do not get stuck
%   \\[5mm]
%   Safety in railway systems
%   \\[5mm]
%   Type systems
%   \\[5mm]
%   \faded{\emph{[Tests: Validation (not Verification)]}}

% \end{slide}


\begin{slide}{Syllabus}
  \frsplitt{
    \begin{itemize}
      \alert{
      \item CSS: a simple language for concurrency
      \begin{itemize}
        \item Syntax
        \item Semantics
        \item Equivalence
      \end{itemize}
      }
      \item  {Timed Automata}
      \begin{itemize}
        \item Syntax
        \item Semantics (composition, Zeno)
        \item Equivalence
        \item {UPPAAL tool}
        \begin{itemize}
           \item Specification
           \item CTL and Verification
         \end{itemize}
      \end{itemize}
    \end{itemize}
  }{
    \begin{itemize}
      \item A simple C-like language
      \begin{itemize}
        \item Syntax
        \item Semantics (operational)
      \end{itemize}
      \item {Hybrid-language}: adding differential equations
      \begin{itemize}
        \item Syntax
        \item Semantics
        \item {Lince tool}
        \begin{itemize}
          \item Specification
          \item Analysis
        \end{itemize}
      \end{itemize}
      \item {Monads}: semantics with computational effects
        % \begin{itemize}
        %   \item Modelling Cyber-physical behaviour with Monads
        %   \item Hybrid Monad
        %   \item Monad Combination
        % \end{itemize}
    \end{itemize}
  }
\end{slide}
\section{Why transition systems?}

\begin{frame}{A Sprinkle of Linguistics}
  During the module we will encounter two linguistic concepts that every programmer
  should know:
  \begin{itemize}
  \item \alert{syntax} - the rules used for determining whether a sentence
    is valid (in a language) or not
  \item \alert{semantics} - the meaning of valid sentences
  \end{itemize}

  \vfill
  \begin{exampleblock}{\exercise Syntax}
      The sentence/program $\mathtt{x := p\, ;q}$ is forbidden by
      the syntactic rules of most programming languages
    \end{exampleblock}
  \begin{exampleblock}{\exercise Semantics}
      The sentence/program $\mathtt{x := 1}$ has the meaning
      ``writes \texttt{1} in the memory address corresponding to \texttt{x}''
  \end{exampleblock}
\end{frame}

\begin{frame}{The need for Semantics in Formal Analysis}

  How can one prove that a program does what is supposed to do if its
  semantics (i.e. its meaning) is not established \emph{a priori} ?

  \begin{exampleblock}{\exercise}
    What is the end result of running
    $x:=2\ ; (x := x + 1 \tikzmark{z1}  \parallel  \tikzmark{z2} x := 0)$ ?
  \end{exampleblock}

  \begin{tikzpicture}[overlay,remember picture,
       box/.style = {rounded corners},
       pin edge={-Stealth,thick, red}]
       \coordinate (z1) at ($({pic cs:z1})+(+0.5ex, 1.5ex)$);
       \coordinate (z2) at ($({pic cs:z2})+(-0.5ex,-0.5ex)$);
       \node[semitransparent, 
             fit=(z1) (z2),
             pin=below:\tiny{parallelism operator}]  {};
   \end{tikzpicture}

   \doExercise{Value of $y$?}{
    $int~x=0~;~ int~\alert{f()}\{ x\texttt{++}; return~x;\}~~int~\alert{g()} \{x\texttt{-{-}}; return~x;\}~~ int~y = \alert{f()}\,\texttt{+}\,\alert{g()};$
   } 

   \vfill
   \underline{Widely used} programming languages \alert{still} lack a formal
   semantics
\end{frame}


\section{Defining Transition System with Functors}

\begin{frame}{Preliminaries pt. I}

  Recalling previous modules \dots

  \vfill
  \begin{definition}[Functor]
    A functor $F$ sends a set $X$ into a new set $FX$ and a function
    $f : X \to Y$ into a new function $F f : F X \to F Y$ such that
    \[
      F (\id) = \id  \hspace{1cm} F(g \comp f) = F g \comp F f
    \]
  \end{definition}

  Fix a set $A$. The following two functors then naturally arise
  \begin{itemize}
  \item product - $X \mapsto A \times X, \hspace{0.2cm} f \mapsto id \times f$
  \item exponential - $X \mapsto X^A, \hspace{0.2cm} f \mapsto (g \mapsto f \comp g)$
  \end{itemize}
\end{frame}

\begin{frame}{Preliminaries pt. II - the List and Powerset functors}
  The list functor - $[X] \mapsto X^\ast, \hspace{0.2cm} [f] \mapsto \tikzmark{x1}
  \prog{map}\ f \tikzmark{x2}$

  \vspace{1cm}
  The powerset functor - almost like the list functor; the difference
  is that we do not look at the order in which elements appear and
  how many times they repeat. Formally,
  \[
    \texttt{P}(X) \mapsto \{ A \mid A \subseteq X \}, \qquad
    \texttt{P}(f) \mapsto (A \mapsto \{ f(a) \mid a \in A \})
  \]

  \begin{exampleblock}{\exercise Powerset on Booleans}
    $\texttt{P}(\mathtt{Bool}) \mapsto \{ \emptyset, \{ \top \}, \{ \bot \}, \{ \top, \bot \} \}$
  \end{exampleblock}
  
  \begin{tikzpicture}[overlay,remember picture,
       box/.style = {rounded corners},
       pin edge={-Stealth,thick, red}]
       \coordinate (x1) at ($({pic cs:x1})+(+0.5ex, 1.5ex)$);
       \coordinate (x2) at ($({pic cs:x2})+(-0.5ex,-0.5ex)$);
       \node[semitransparent, 
             fit=(x1) (x2),
             pin=below:\tiny{applies $f$ to every element of a given list}]  {};
   \end{tikzpicture}

\end{frame}


\begin{frame}{A (Generalised) Notion of a Transition System}
  \begin{definition}[Transition system]
    Let $F$ be a functor. An $F$-transition system is 
    a map $X \to F X$
  \end{definition}

  \vfill
  Some famous examples of $F$-transition systems
  \begin{itemize}
  \item Moore machine - $X \to \Act \times X$
  \item Deterministic automata - $X \to \mathtt{Bool} \times X^{\Act}$
  \item Non-deterministic automata - $X \to \mathtt{Bool} \times \tikzmark{a1} \Pow(X)^{\Act} \tikzmark{a2}$
  \item Markov chain - $X \to \tikzmark{b1} \Dist(X) \tikzmark{b2}$
  \end{itemize}

  \begin{tikzpicture}[overlay,remember picture,
     box/.style = {rounded corners},
     pin edge={-Stealth,thick, red}]
     \coordinate (a1) at ($({pic cs:a1})+(+0.5ex, 1.5ex)$);
     \coordinate (a2) at ($({pic cs:a2})+(-0.5ex,-0.5ex)$);
     \node[semitransparent, 
           fit=(a1) (a2),
           pin=below:\tiny{Powerset functor}]  {};
  \end{tikzpicture}
  \begin{tikzpicture}[overlay,remember picture,
     box/.style = {rounded corners},
     pin edge={-Stealth,thick, red}]
     \coordinate (b1) at ($({pic cs:b1})+(+0.5ex, 1.5ex)$);
     \coordinate (b2) at ($({pic cs:b2})+(-0.5ex,-0.5ex)$);
     \node[semitransparent, 
           fit=(b1) (b2),
           pin=below:\tiny{Distribution functor}]  {};
  \end{tikzpicture}

\end{frame}


\begin{slide}{Exercise}
  \centering
  \doExercise{Formalise as an F-transition system}{
    \centering 
    \wrap{\begin{tikzpicture}[aut]
      \node[st,initial above]  (s_0) {$A_0$};
      \node[st,final] (s_1) [right of=s_0] {$A_1$};
      \path[->]
        (s_0) edge  node {a}  (s_1)
        (s_0) edge[loop left] node {b}  ()
        (s_1) edge[loop right] node {c}  (s_0);
    \end{tikzpicture}}
    ~~~~~~~~~~
    \wrap{\begin{tikzpicture}[aut]
      \node[st,initial above]  (s_0) {$B_0$};
      \node[st] (s_1) [right of=s_0] {$B_1$};
      \path[->]
        (s_0) edge  node {a}  (s_1)
        (s_0) edge[loop left] node {a}  ()
        (s_1) edge[loop right] node {b}  (s_0);
    \end{tikzpicture}}
    \\[6mm]
    \wrap{\begin{tikzpicture}[aut]
      \node[st,initial left]  (s_0) {$C_0$};
      \node[st] (s_1) [right of=s_0] {$C_1$};
      \node[st] (s_2) [right of=s_1] {$C_2$};
      \node (s_3) [right of=s_2] {$\cdots$};
      \node[below] at(s_0.south){$a$};
      \node[below] at(s_1.south){$a$};
      \node[below] at(s_2.south){$a$};
      \path[->]
        (s_0) edge (s_1)
        (s_1) edge  (s_2)
        (s_2) edge  (s_3);
    \end{tikzpicture}}
  }
\end{slide}


\begin{frame}{Our First encounter with Coalgebra}

  Indeed the idea of working at the level of

  \begin{center}
    \alert{Functors as Transition Types}
  \end{center}

  is a very fruitful one; and which we only barely
  grasped (yet) ---

  in essence, it provides a \alert{universal theory} of transition
  systems that can be instantiated to most kinds of transition system
  we will encounter in our life
\end{frame}





\section{Process algebra}

%-------------------------------------------------------------------------------
\begin{slide}{Process algebras}
\small

\begin{block}{Sequential CCS - Syntax}
\begin{equation*}
% \mathcal{P} ~\ni~ P,Q\; ::=\; K ~~|~~ \alpha.P ~~|~~ \sum_{i\in I} P_i
%         ~~|~~ \crn P f \faded{~~|~~ P|Q ~~|~~ P\backslash L}
\mathcal{P} ~\ni~ P,Q\; ::=\; K ~~|~~ \alpha.P ~~|~~ P + Q ~~|~~ \cnil
        ~~|~~ \crn P f  ~~|~~ P\backslash L \faded{~~|~~ P|Q}
\end{equation*}
%
where
\\- $\alpha \in \Act %\faded{\cup \overline\Act \cup \set{\tau}}
    \cup \set{\tau}$~ is an \structure{action}
\\- $K$ s a collection of \structure{process} names or process constants
% \\- $I$ is an indexing set
\\- $L \subseteq \Act %\faded{\cup \overline{\Act}}
    $ is a set of \structure{labels}
\\- $f$ is a function that \structure{renames} actions s.t. $f(\tau) = \tau$ % and $f(\overline{a}) = \overline{f(a)}$
\\- \alert{notation:}
% \\~~~~~$\cnil = \sum_{i\in\emptyset}P_i$
% \\~~~~~$P_1+P_2 = \sum_{i\in\set{1,2}}P_i$
\\~~~~~$[f] = [a_1\mapsto b_1,\ldots,a_n \mapsto b_n]$
\end{block}
\end{slide}

%-------------------------------------------------------------------------------

\begin{slide}{Process algebras}
\small

\begin{block}{Syntax}
\begin{equation*}
% \mathcal{P} ~\ni~ P,Q\; ::=\; K ~~|~~ \alpha.P ~~|~~ \sum_{i\in I} P_i
%         ~~|~~ \crn P f \faded{~~|~~ P|Q ~~|~~ P\backslash L}
\mathcal{P} ~\ni~ P,Q\; ::=\; K ~~|~~ \alpha.P ~~|~~ P + Q ~~|~~ \cnil
        ~~|~~ \crn P f  ~~|~~ P\backslash L \faded{~~|~~ P|Q}
\end{equation*}
\end{block}

%\setcounter{equation}{0}
\begin{exampleblock}{\exercise Which are NOT syntactically correct? Why?}
\begin{columns}
  \column{0.38\textwidth}
  \begin{align}
    & a.b.A+B\\&
    (a.\cnil + b.A) \backslash \set{a,b,c}\\&
    (a.\cnil + b.A) \backslash \set{a,\tau}\\& % no
    a.B+[b\mapsto a]\\& % no
    \tau.\tau.B + \cnil
  \end{align}

  \column{0.45\textwidth}
  \begin{align} &
    a.(a + b).A\\&
    (a.B + b.B)[a\mapsto a,\tau\mapsto b]\\&
    (a.B + \tau.B)[b\mapsto a,a\mapsto a]\\& % no
    % (a.b.A + \ainv a.\cnil)|B\\&
    (a.b.A + b.\cnil).B\\&
    (a.b.A + b.\cnil)+B
    % \\&
    % (\cnil | \cnil) + \cnil
  \end{align}
\end{columns}
\end{exampleblock}

\end{slide}

%-------------------------------------------------------------------------------

\exerciseBack
\begin{slide}{CCS semantics - building a transition system}
\small 
Every $P$ yields a transition system $X \to {\alert{???}}$ with transitions prescribed by the rules below.

\vspace*{-1mm}
\centering
\newcommand{\msep}{~~~~~~}

\typerule{act}{\shrk}{\alpha.P \trans\alpha P}
\msep
\typerule{sum-1}{P_1 \trans\alpha P'_1}{P_1 + P_2 \trans\alpha P'_1} %$\faded{j\in I}$
\msep
\typerule{sum-2}{P_2 \trans\alpha P'_2}{P_1 + P_2 \trans\alpha P'_2} %$\faded{j\in I}$
\\[3mm]
\typerule{res}{P\trans\alpha P'}{\crt P L\trans\alpha \crt{P'}L}~${\alpha%,\ainv{\alpha}
                                                                  \notin L}$
\msep
\typerule{rel}{P\trans\alpha P'}{\crn P f\trans{f(\alpha)} \crn{P'} f}
% \\[3mm]
% \typerule{com1}{P\trans\alpha P'}{P|Q\trans\alpha P'|Q}
% \msep
% \typerule{com2}{Q\trans\alpha Q'}{P|Q\trans\alpha P|Q'}
% \msep
% \typerule{com3}{P\trans a P' \quad Q\trans{\ainv{a}} Q'}{P|Q\trans\tau P'|Q'}
\\[6mm]

\only<1>{
  \begin{itemize}
    \item \structure{Initial states:} the process being translated
    \item \structure{Final states:} all states are final
    \item \alert{Language:} possible sequence of actions of a process
  \end{itemize}
}
\pause

\begin{exampleblock}{\exercise Build a derivation tree to prove the transitions below}
  \vspace*{-2mm}
  \begin{enumerate}
    \item $(a.A + b.B) ~\trans{b}~ B$
    \item $(a.b.A + (b.a.B + c.a.C)) ~\trans{b}~ a.B$
    \item $((a.B + b.A)[a \mapsto c])\backslash\{a,b\} ~\trans{c}~ (B[a\mapsto c])\backslash\{a,b\}$
  \end{enumerate}
\end{exampleblock}

\end{slide}

\begin{slide}{Exercise}
  \begin{exampleblock}{\exercise Draw the automata}
  % \vspace*{-4mm}
  \begin{align*}
    CM &= \mathsf{coin.coffee}.CM
    \\
    CS &= \mathsf{pub.(coin.coffee.CS + coin.tea.CS)}
    % CM &= \mathsf{coin.\ainv{coffee}}.CM
    % \\
    % CS &= \mathsf{\ainv{pub}.\ainv{coin}.coffee}.CS
    % \\
    % SmUni &= \crt{(CM|CS)}{\mathsf{\set{coin,coffee}}}
  \end{align*}
\end{exampleblock}

\doExercise{What is the language of the process $A$?}{
\centering
  \vspace*{-5mm}
  \begin{align*}
    A &= \mathsf{goLeft.A + goRight.B}
    \\
    B &= \mathsf{rest.\cnil}
  \end{align*}
}
\end{slide}


\begin{slide}{Exercise}
  \centering

  \includegraphics[width=1.0\textwidth]{images/coffee-flow.png}

  \doExercise{Write the process of the flowchart above}{
        $P ~~=~ \mf{powerOn} . Q$\\
        $Q ~~=~ \mf{selMocha}.\mf{addChocolate}.Mk + \mf{selLatte}.Mk + \ldots $\\
        $Mk ~=~ \mf{addMilk}\ldots$  
      ~\\[0.35\textheight]
  }

\end{slide}



% ----------------------------------------

\section{Concurrent Process algebra}


\begin{slide}{Overview}

\begin{block}{Recall}
\begin{enumerate}
  \item Non-deterministic Finite Automata ($X \to \faded{\mathtt{Bool}} \times \mathtt{P}(X)^{\Act}$):
  \wrap{\small\begin{tikzpicture}[aut]
  \node[st,initial]  (q1) {$q_1$};
  \node[st,final] (q2) [right of=q1]  {$q_2$};
  \path[->] (q1) edge node {a} (q2) (q2) edge[loop right] node {b} ();
\end{tikzpicture}}
  \item (Sequential) Process algebra:
    $P = a.Q ~~~~ Q=b.Q$

  \item Meaning of \structure{(2)} using \structure{(1)}
\end{enumerate}  
\end{block}

\begin{block}{Still missing}
\begin{itemize}
  \item \alert{\textbf{Interaction}} between processes
  % \item \emph{Interaction \alert{diagrams}} vs. \emph{interacting \alert{processes}}
  \item Enrich \structure{(2)} and \structure{(3)}
\end{itemize}
\end{block}

\end{slide}

%-------------------------------------------------------------------------------
\begin{slide}{Process algebras}
\small

\begin{block}{CCS - \alert{Updated} Syntax}
\begin{equation*}
% \mathcal{P} ~\ni~ P,Q\; ::=\; K ~~|~~ \alpha.P ~~|~~ \sum_{i\in I} P_i
%         ~~|~~ \crn P f \transp{~~|~~ P|Q ~~|~~ P\backslash L}
\mathcal{P} ~\ni~ P,Q\; ::=\; K ~~|~~ \alpha.P ~~|~~ P + Q ~~|~~ \cnil
        ~~|~~ \crn P f  ~~|~~ P\backslash L ~~|~~ \alert{P|Q}
\end{equation*}
%
where
\\- $\alpha \in \Act \alert{\cup \overline\Act}
    \cup \set{\tau}$~ is an action
\\- $K$ s a collection of process names or process constants
\\- $L \subseteq \Act %\transp{\cup \overline{\Act}}
    $ is a set of labels
\\- $f$ is a function that renames actions s.t. $f(\tau) = \tau$  \alert{and $f(\overline{a}) = \overline{f(a)}$}
\\- notation:
% \\~~~~~$\cnil = \sum_{i\in\emptyset}P_i$
% \\~~~~~$P_1+P_2 = \sum_{i\in\set{1,2}}P_i$
\\~~~~~$[f] = [a_1\mapsto b_1,\ldots,a_n \mapsto b_n]$~~~~~where \alert{$a_i,b_i \in N \cup \set{\tau}$}
\end{block}
\end{slide}

%-------------------------------------------------------------------------------

\begin{slide}{Process algebras}
\small

\begin{block}{Syntax}
\begin{equation*}
% \mathcal{P} ~\ni~ P,Q\; ::=\; K ~~|~~ \alpha.P ~~|~~ \sum_{i\in I} P_i
%         ~~|~~ \crn P f \transp{~~|~~ P|Q ~~|~~ P\backslash L}
\mathcal{P} ~\ni~ P,Q\; ::=\; K ~~|~~ \alpha.P ~~|~~ P + Q ~~|~~ \cnil
        ~~|~~ \crn P f  ~~|~~ P\backslash L ~~|~~  \alert{P|Q}
\end{equation*}
\end{block}

%\setcounter{equation}{0}
\begin{exampleblock}{\exercise Which are syntactically correct?}
\begin{columns}
  \column{0.38\textwidth}
  \begin{align}
    & a.\ainv b.A+B\\&
    (a.\cnil + \ainv a.A) \backslash \set{\ainv{a},b}\\&
    (a.\cnil + \ainv a.A) \backslash \set{a,\tau}\\& % no
    (a.\cnil + \ainv{\tau}.A) \backslash \set{a}\\& % no
    % a.B+[b\mapsto a]\\& % no
    \tau.\tau.B + \ainv a.\cnil %\\&
    % a.(a + b).A
    \\&
    (\cnil | \cnil) + \cnil
  \end{align}

  \column{0.45\textwidth}
  \begin{align} &
    (a.B + b.B)[a\mapsto a,\tau\mapsto b]\\&
    (a.B + \tau.B)[b\mapsto a,b\mapsto a]\\& % no
    (a.B + b.B)[a\mapsto b,b\mapsto \ainv a]\\& % no
    (a.b.A + \ainv a.\cnil)|B\\&
    (a.b.A + \ainv a.\cnil).B\\&
    (a.b.A + \ainv a.\cnil)+B
  \end{align}
\end{columns}
\end{exampleblock}

\end{slide}

%-------------------------------------------------------------------------------

\begin{slide}{CCS semantics - building an NFA}
\small 
\centering
\newcommand{\msep}{~~~~~~}
\vspace*{-2mm}

\typerule{act}{\shrk}{\alpha.P \trans\alpha P}
\msep
\typerule{sum-1}{P_1 \trans\alpha P'_1}{P_1 + P_2 \trans\alpha P'_1} %$\transp{j\in I}$
\msep
\typerule{sum-2}{P_2 \trans\alpha P'_2}{P_1 + P_2 \trans\alpha P'_2} %$\transp{j\in I}$
\\[2mm]
\typerule{res}{P\trans\alpha P'}{\crt P L\trans\alpha \crt{P'}L}~$\transp{\alpha,\ainv{\alpha}
                                                                  \notin L}$
\msep
\typerule{rel}{P\trans\alpha P'}{\crn P f\trans{f(\alpha)} \crn{P'} f}
\\[2mm]
\alert{
\typerule{com1}{P\trans\alpha P'}{P|Q\trans\alpha P'|Q}
\msep
\typerule{com2}{Q\trans\alpha Q'}{P|Q\trans\alpha P|Q'}
\msep
\typerule{com3}{P\trans a P' \quad Q\trans{\ainv{a}} Q'}{P|Q\trans\tau P'|Q'}
}
\\[2mm]
\pause

\begin{exampleblock}{\exercise Draw the transition systems}
  \exerciseBack
  \vspace*{-5mm}
  \begin{align*}
    % CM &= \mathsf{coin.coffee}.CM
    % \\
    % CS &= \mathsf{pub.(coin.coffee.CS + coin.tea.CS)}
    CM &= \mathsf{coin.\ainv{coffee}}.CM
    \\
    CS &= \mathsf{pub.\ainv{coin}.coffee}.CS
    \\
    SmUni &= \crt{(CM|CS)}{\mathsf{\set{coin,coffee}}}
  \end{align*}
  \vspace*{-7mm}
\end{exampleblock}
\end{slide}
\exerciseAdd



\begin{slide}{Exercises}
  \doExercise{Let $A=b.a.B$. Show that:}{
    \vspace*{-8mm}
    \begin{enumerate}
      \item $(A ~|~ \overline{b}.\cnil)\backslash \{b\}~\trans\tau~ (a.B ~|~ \cnil)\backslash\{b\}$
      \item $(A ~|~ b.a.B) + ((b.A)[b\mapsto a]) ~\trans{a}~ A[b \mapsto a]$
    \end{enumerate}
  }  

  \doExercise{Draw the NFAs $A$ and $D$}{
    \vspace*{-8mm}

    \begin{columns}
    \col[0.45]{      
    \begin{align*}
      A &= x.B+x.x.C\\
      B &= x.x.A+y.C\\
      C &= x.A      
    \end{align*}
    }
    \col[0.45]{      
    \begin{align*}
      D &= x.x.x.D + x.E\\
      E &= x.F+y.F\\
      F &= x.A      
    \end{align*}
    }
    \end{columns}
  }  

\end{slide}


\section{Observational Equivalence}

\begin{slide}{Overview}

\begin{block}{Recall}
\begin{enumerate}
  \item F-transition systems, e.g., Non-deterministic Finite Automata:
  \wrap{\small\begin{tikzpicture}[aut]
  \node[st,initial]  (q1) {$q_1$};
  \node[st,final] (q2) [right of=q1]  {$q_2$};
  \path[->] (q1) edge node {a} (q2) (q2) edge[loop right] node {b} ();
\end{tikzpicture}}
  \item Process algebra:
    $P = a.Q ~~~~ Q=b.Q ~~~~ P|Q$

  \item Interaction between processes
  
  \item Meaning of CCS using transition systems

  % \item Equivalence relations ((bi)simulations)
\end{enumerate}  
\end{block}

\begin{block}{Still missing}
\begin{itemize}
  \item When is a process $P$ \alert{equivalent} to a process $Q$?
  \item When can a process $P$ be \alert{safely replaced} by a process $Q$?
  % \item When can a sequence of interactions be \alert{safely implemented} as interacting components?
\end{itemize}
\end{block}
\end{slide}


\begin{frame}{Observational Equivalence Informally}

  Two programs are \structure{observationally equivalent} if it is impossible to \alert{observe any
  difference} in their \alert{behaviour}

  
  \vfill
  Here behaviour is described in terms of transition systems

  \dots\ and therefore behaviour/equivalence needs to be pinned
  down to them
\end{frame}

\section{EQ1 -- Language equivalence}

%----------------------------------------------------------------------------------
\begin{slide}{Language  equivalence}
\small

\begin{block}{Definition}
Two automata $A, B$ are \alert{language equivalent} iff  $ L_A =  L_B$\\
(i.e. if they can perform the same finite sequences of transitions)
\end{block}
~\\

\begin{example}
  \centering
  % Requires \usepackage{graphicx}
  \includegraphics[width=9cm]{./images/alarm3.pdf}
\end{example}


\alert{Language equivalence} applies  when one can neither interact with a system, nor distinguish a slow system from one that has come to a stand still.
\end{slide}


% \begin{slide}{Exercise}
%   \centering
%   \doExercise{Consider the system}{
%   \centering
%   \begin{tikzpicture}
%       \node(p){$p$}; \node[right=of p](p1){$p_1$}; \node[right=of p1](p2){$p_2$};
%       \draw[->] (p)edge[bend right] node[below]{a} (p1)
%                 (p1)edge[bend right] node[above]{b}(p)
%                 (p2)edge node[above]{d} (p1)edge[loop right] node[right]{c}(p1) ;
%   \end{tikzpicture}
%   \begin{enumerate}
%     \item Formalise the system as a tuple $(S,Act,\to)$
%     \item List 4 different traces from state (p2).
%   \end{enumerate}
%   }
% \end{slide}


\begin{slide}{Exercise}
  \centering
  \doExercise{Find pairs of automata with the same language}{
    \centering 
    \wrap{\begin{tikzpicture}[aut]
      \node[st,initial above,final]  (s_0) {$A_0$};
      \node[st] (s_1) [right of=s_0] {$A_1$};
      \path[->]
        (s_0) edge  node {a}  (s_1)
        (s_0) edge[loop left] node {a}  ()
        (s_1) edge[loop right] node {b}  (s_0);
    \end{tikzpicture}}
    ~~~~~~~~~~
    \wrap{\begin{tikzpicture}[aut]
      \node[st,initial above,final]  (s_0) {$B_0$};
      \node[st,final] (s_1) [right of=s_0] {$B_1$};
      \path[->]
        (s_0) edge  node {a}  (s_1)
        (s_0) edge[loop left] node {a}  ()
        (s_1) edge[loop right] node {a}  (s_0);
    \end{tikzpicture}}
    ~~~~~~~~~
    \wrap{\begin{tikzpicture}[aut]
      \node[st,initial above]  (s_0) {$C_0$};
      \node[st,final] (s_1) [right of=s_0] {$C_1$};
      \path[->]
        (s_0) edge  node {a}  (s_1)
        (s_0) edge[loop left] node {a}  ()
        (s_1) edge[loop right] node {b}  (s_0);
    \end{tikzpicture}}
    ~~~~~~~~~
    \wrap{\begin{tikzpicture}[aut]
      \node[st,initial above]  (s_0) {$D_0$};
      \node[st,final] (s_1) [right of=s_0] {$D_1$};
      \path[->]
        (s_0) edge  node {a}  (s_1)
        (s_0) edge[loop left] node {a}  ()
        (s_1) edge[loop right] node {a}  (s_0);
    \end{tikzpicture}}
    ~~~~~~~~~
    \wrap{\begin{tikzpicture}[aut]
      \node[st,initial above]  (s_0) {$E_0$};
      \node[st,final] (s_1) [right of=s_0] {$E_1$};
      \path[->]
        (s_0) edge[bend left]  node {a}  (s_1)
        (s_1) edge[bend left]  node {a}  (s_0)
        (s_1) edge[loop right] node {a}  (s_0);
    \end{tikzpicture}}

  }
\end{slide}

\begin{slide}{Exercise}
  \doExercise{Check if the processes are language equivalent}{
    \begin{align*}
      P &= coin.(\ainv{coffee}.P + \ainv{tea}.P)
      &
      Q &= coin.\ainv{coffee}.Q + coin.\ainv{tea}.Q
    \end{align*}
  }

\end{slide}


\section{EQ2 -- Similarity}

%----------------------------------------------------------------------------------
\begin{slide}{Simulation}
\begin{flushright}
the quest for a \alert{behavioural equality}:\\
able to identify states that cannot be distinguished by any \alert{realistic} form of  observation
\end{flushright}
~\\

\small
\begin{block}{Simulation}
% \caixa
{A state $q$ \alert{simulates} another state $p$
\alert{if}\\
every transition from $q$ is corresponded by a transition from $p$
\alert{and}\\
this capacity is kept along the whole life of the system to which state space $q$ belongs to.}
\end{block}
\end{slide}

%----------------------------------------------------------------------------------
\begin{slide}{Simulation of NFA ($X \to \mathtt{P}(X)^{\Act}$)}
\small

\begin{block}{Definition}
% Given  $\pair{S_1, \Act, \rra_1}$  and $\pair{S_2, \Act, \rra_2}$
%  over $\Act$ \faded{(ignoring initial and final states)}
Given NFA $A_1$ and $A_2$ over \Act\ with states $S_1$ and $S_2$ respectively,
a relation \structure{$R \subseteq S_1 \times S_2$} is a \structure{simulation} iff,
for all $\pair{p,q} \in \structure{R}$ and $a \in \Act$,

\begin{align*}
%\text{\gold{(1)}}\; \;  & p \downarrow_1 \;  \imp\; q \downarrow_2\\ &\\
\text{\alert{(1)}}\; \;  & p \rtran{a}_1 p'\;  \imp\; \rcb{\exists}{q'}{q' \in S_2}{q \rtran{a}_2 q'\, \land\, \pair{p',q'} \in \structure{R}}   
\end{align*}
\vspace{0mm}

\begin{equation*}
\xymatrix{
p \ar[d]^-{a} & \hspace{-1.5cm} \structure{R}  & \hspace{-2.7cm} q 
  & \!\! \text{\raisebox{-10mm}{\Huge $\imp$}} \! \! &  &  &  q \ar[d]^-{a} \\
p'           &   &           &                                   & p'\hspace{-2.7cm} &  \structure{R} \hspace{-1.5cm} &  q'
}
\end{equation*}

\end{block}
\end{slide}

%----------------------------------------------------------------------------------
\begin{slide}{Example}

\begin{exampleblock}{\exercise Find simulations}
\begin{equation*}
\xymatrix{
& q_1  \ar[r]^{d} & q_2 &       &        &                              p_2\\
q_0 \ar[ru]^{a} \ar[rd]_{a} &  & & p_0 \ar[r]^{a} & p_1 \ar[ru]^{d} \ar[rd]_{e} & \\
& q_4  \ar[r]_{e} & q_3 &       &        &                              p_3\\
}
\end{equation*}
\end{exampleblock}

\vspace{0.2cm}
\visible<2->{\exerciseBack\begin{equation*}
q_0 \lesssim p_0 \text{\hspace{0.5cm} cf. \hspace{0.3cm}} 
\set{\pair{q_0,p_0}, \pair{q_1,p_1},\pair{q_4,p_1},\alert{\ldots}} %\pair{q_2,p_2},\pair{q_3,p_3}}
\end{equation*}}
\end{slide}

\exerciseAdd


%----------------------------------------------------------------------------------
\begin{slide}{Similarity}
\small

\begin{block}{Definition}
\centering
\[p \lesssim q\; \equiv\; \rcb{\exists}{R}{}{R\; \text{is a simulation and}\; \pair{p,q} \in R} 
\]
\emph{We say \alert{$p$ is simulated by $q$}.}
\end{block}

% \begin{block}{Automata simulation}
%   \[\pair{S_1,\set{s_1},\dda_1,\rra_1} \lesssim \pair{S_2,\set{s_2},\dda_2,\rra_2}\; \equiv\; s_1 \lesssim s_2 
% \]
% \end{block}

\begin{block}{Lemma}
The similarity relation is a preorder\\
\faded{(ie, reflexive and transitive)}
\end{block}
\end{slide}



%-------------------------

\section{EQ3 -- Bisimilarity}
%----------------------------------------------------------------------------------
\begin{slide}{Bisimulation}
\small

\begin{block}{Definition}
% Given  $\pair{S_1, \Act,  \rra_1}$  and $\pair{S_2, \Act, \rra_2}$ over $\Act$,
Given NFA $A_1$ and $A_2$ over \Act\ with states $S_1$ and $S_2$ respectively,
relation $R \subseteq S_1 \times S_2$ is a \structure{bisimulation} iff both $R$ and its converse $\aconv{R}$
are simulations.

I.e.,
whenever $\pair{p,q} \in R$ and $a \in \Act$,

\begin{align*}
%\text{\gold{(1)}}\; \;  &  p \downarrow_1 \;  \dimp\; q \downarrow_2\\ &\\
\text{\alert{(1)}}\; \;  & p \rtran{a}_1 p'\; \imp\; \rcb{\exists}{q'}{q' \in S_2}{q \rtran{a}_2 q'\, \land\, \pair{p',q'} \in R}   \\
\text{\alert{(2)}}\; \;  & q \rtran{a}_2 q'\; \imp\; \rcb{\exists}{p'}{p' \in S_1}{p \rtran{a}_1 p'\, \land\, \pair{p',q'} \in R}   
\end{align*}

%\begin{equation*}
%\xymatrix{
%p \ar[d]^-{a} & \hspace{-1.5cm} R  & \hspace{-2.7cm} q
%    & \! \! \text{\raisebox{-10mm}{\Huge $\Leftrightarrow$}} \! \! &   &  
%    &  q \ar[d]^-{a} \\
%p'           &   &           &                                   & p'\hspace{-2.7cm} &  R \hspace{-1.5cm} &  q'
%}
%\end{equation*}
\centering
\begin{tikzpicture}[%
    every edge/.style={draw, thick,-latex',shorten >= 2pt}]
  \tikzstyle{l} = [auto,inner sep=1pt]
  % the square
  \node(p){$p$};\node[right=2.3 of p](q2){$q$};
  \node[below=1 of p](p'){$p'$};\node(q')at(p'-|q2){$q'$};
  % the R parts
  \node[right=0 of p](R1){$R$};\node[right=0 of R1](q){$q$};
  \node[left=0 of q'](R2){$R$};\node[left=0 of R2](p'2){$p'$};
  % the arrows
  \draw[->] (p)edge node[l]{$a$}(p') (q2)edge node[l]{$a$}(q');
  \node at ($(p)!.5!(q')$){{\Large $\Rightarrow$}};
  %%%%%%%
  % the square
  \node[right=2.3 of q2](pp){$p$};\node[right=2.3 of pp](q2){$q$};
  \node[below=1 of pp](p'){$p'$};\node(q')at(p'-|q2){$q'$};
  % the R parts
  \node[right=0 of p'](R1){$R$};\node[right=0 of R1](q){$q'$};
  \node[left=0 of q2](R2){$R$};\node[left=0 of R2](p'2){$p$};
  % the arrows
  \draw[->] (pp)edge node[l]{$a$}(p') (q2)edge node[l]{$a$}(q');
  \node at ($(pp)!.5!(q')$){{\Large $\Leftarrow$}};

\end{tikzpicture}

\end{block}
\end{slide}



%----------------------------------------------------------------------------------
\begin{slide}{Examples}

\doExercise{Find bisimulations that include $\pair{q_1,m}$}{
\begin{equation*}
\xymatrix{
& q_1  \ar[ld]_{a}  \ar[rd]^{a} & & & & m \ar[d]_{a}\\
q_2 \ar[rr]^{c}  && q_3 \ar@(ur,dr)[]^{c}  & & & n\ar@(ur,dr)[]^{c}
}
\end{equation*}
% R = \enset{(q_1,m), (q_2, n), (q_3,n) }
}

\doExercise{Find bisimulations that include $\pair{q_1,h}$}{

\begin{equation*}
\xymatrix{
q_1 \ar[r]^{a} & q_2  \ar[r]^{a}  & q_3  \ar[r]^{a} & \cdots & & h\ar@(ur,dr)[]^{a}  \\
}
\end{equation*}
% R = \setdef{q_i, h}{i \geq 1}  
}

\end{slide}


%----------------------------------------------------------------------------------
\begin{slide}{Exercises}

\doExercise{Check if there is a bisimulation that include $\pair{q_1,p_1}$}{
\vspace*{-3mm}
\begin{equation*}
\xymatrix{
& q_1  \ar[ld]_{a}  \ar[rd]^{a} & & & & p_1  \ar[d]^{a} \\
q_2 \ar[d]^{c}  && q_3  \ar[d]^{c}  & && p_2  \ar[ld]_{c}  \ar[rd]^{c} & \\
q_4 & & q_5 & &p_4 & & p_5\\
% & q_1  \ar[ld]_{a}  \ar[rd]^{a} & & & & p_1  \ar[d]^{a} \\
% q_2 \ar[d]^{\rdb{b}}  && q_3  \ar[d]^{c}  & && p_2  \ar[ld]_{\rdb{b}}  \ar[rd]^{c} & \\
% q_4 & & q_5 & &p_4 & & p_5
}
\end{equation*}
}
\end{slide}


\begin{slide}{Exercises}
\doExercise{Check if there is a bisimulation that include $\pair{q_1,p_1}$}{
\vspace*{-3mm}
\begin{equation*}
\xymatrix{
% & q_1  \ar[ld]_{a}  \ar[rd]^{a} & & & & p_1  \ar[d]^{a} \\
% q_2 \ar[d]^{c}  && q_3  \ar[d]^{c}  & && p_2  \ar[ld]_{c}  \ar[rd]^{c} & \\
% q_4 & & q_5 & &p_4 & & p_5\\
& q_1  \ar[ld]_{a}  \ar[rd]^{a} & & & & p_1  \ar[d]^{a} \\
q_2 \ar[d]^{\rdb{b}}  && q_3  \ar[d]^{c}  & && p_2  \ar[ld]_{\rdb{b}}  \ar[rd]^{c} & \\
q_4 & & q_5 & &p_4 & & p_5
}  
\end{equation*}
}

\doExercise{Check if there is a bisimulation that include $\pair{P,Q}$}{
  \vspace*{-5mm}
  \begin{align*}
      P &= coin.(\ainv{coffee}.P + \ainv{tea}.P)
      &
      Q &= coin.\ainv{coffee}.Q + coin.\ainv{tea}.Q
  \end{align*}
}

\end{slide}



% Definition
% p∼q ≡ ⟨∃R :: Risabisimulationand⟨p,q⟩∈R⟩
%----------------------------------------------------------------------------------
\begin{slide}{Bisimilarity}
\small

\begin{block}{Definition}
\centering
\[p \sim q\; \equiv\; \rcb{\exists}{R}{}{R\; \text{is a bisimulation and}\; \pair{p,q} \in R} 
\]
\emph{We say \alert{$p$ is bisimilar to $q$}.}
\end{block}


\begin{block}{Lemma}
\centering
Two processes $P$ and $Q$ are bisimilar if there is a bisimulation that includes $\pair{P,Q}$.
\end{block}

\begin{block}{Lemma}
The bisimilarity relation is an equivalence relation\\
\faded{(ie, symmetric, reflexive and transitive)}
\end{block}


% \begin{block}{Automata simulation}
%   \[\pair{S_1,\set{s_1},\dda_1,\rra_1} \lesssim \pair{S_2,\set{s_2},\dda_2,\rra_2}\; \equiv\; s_1 \lesssim s_2 
% \]
% \end{block}

% \begin{block}{Lemma}
% The similarity relation is a preorder\\
% \faded{(ie, reflexive and transitive)}
% \end{block}
\end{slide}



\section{Generalising Observational Equivalences}

\begin{frame}{$F$-Transition Systems and Observational \underline{Equivalence}}

  \begin{definition}
    Fix a functor $F$ and consider two transition systems
    $f : X \to F X$ and $g : Y \to F Y$. Two states $x \in X$, $y \in Y$
    are \alert{observationally equivalent if}
      \begin{itemize}
         \item there exists \structure{a relation}
      $\structure{R} \subseteq X \times Y$ with $(x,y) \in R$ and
         \item there exists \structure{a transition system} $\structure{b} : R \to FR$ such that the diagram below commutes
       \end{itemize} 
    \[
      \xymatrix{
        X \ar[d]_{f} & \ar[l]_{\pi_1}  \structure{R} \ar[r]^{\pi_2} \ar[d]_{\structure{b}} & Y \ar[d]^{g} \\
        F X  & \ar[l]^{F \pi_1} \structure{FR} \ar[r]_{F \pi_2} & F Y
        }
    \]
    If such is the case we write \alert{$x \sim y$}  
  \end{definition}
\end{frame}

\begin{frame}{Observational Equivalence for Moore Automata}
  Given $\pv{o_1}{n_1} : X \to A \times X$ and
  $\pv{o_2}{n_2} : Y \to A \times Y$ we obtain from the previous slide 
  that $x \sim y$ iff
  \begin{itemize}
  \item $o_1(x) = o_2(y)$
  \item $n_1(x) \sim n_2(y)$
  \end{itemize}
\end{frame}

\begin{frame}{Observational Equivalence for Labelled Transition Systems}
  Recall that we used systems of type $X \to \Pow(X)^\Act$ for establishing the
  \alert{semantics} of \alert{CCS processes}. This means that \dots

  notions of observational behaviour/equivalence for such transition systems
  directly impact our concurrent language

  \vfill Given $\overline{t_1} : X \to \Pow(X)^\Act$ and
  $\overline{t_2} : Y \to \Pow(Y)^\Act$, $x \sim y$ iff for all $l \in \Act$
  \begin{itemize}
  \item $\forall x' \in t_1(x,n).\  \exists y' \in t_2(y,n).\ x' \sim y'$
  \item $\forall y' \in t_2(y,n).\ \exists x' \in t_1(x,n).\ x' \sim y'$
  \end{itemize}
\end{frame}


% \begin{frame}{Is Observational Equivalence a Good Notion of Equivalence?}

%   \begin{block}{Coinduction Principle}
%       Two states $x,y$ are observationally \alert{equivalent} iff they produce the
%       \alert{same} observational behaviour
%   \end{block}
  
% \end{frame}

\end{document}
