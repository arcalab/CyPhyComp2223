%\documentclass[handout]{beamer}
\documentclass[aspectratio=169]{beamer}
\usepackage{etex} % fixes new-dimension error


\usepackage{graphicx,amsmath}
\usepackage{stmaryrd} % cf. interleave
\usepackage{booktabs}
\usepackage{amscd}
\usepackage{multicol}
\usepackage[absolute,overlay]{textpos}
\usepackage{alltt}
\usepackage{proof}
\usepackage{subcaption}
\usepackage{tikz}
\usepackage{tikz-cd}
\usepackage[new]{old-arrows}
\usepackage[all]{xy}
\usepackage{pgfplots}
\usepackage{textcomp}
\usepackage{listings}

\input{macros/macros}
\input{macros/beamerconf}
% \input{macros/preamble}


%\def\larrow#1#2#3{\xymatrix{ #3 & #1 \ar[l] _-{#2} }}
\def\larrow#1#2#3{\xymatrix{ #3 & #1 \ar[l] _--{#2} }}
\def\rarrow#1#2#3{\xymatrix{ #1 \ar[r]^-{#2} & #3 }}
\def\arLaw#1#2#3#4#5{
\xymatrix{
        #1      \ar@/^1pc/[rr]^-{#4} &
        #5 &
        #2      \ar@/^1pc/[ll]^-{#3}
}}
\def\arLeq#1#2#3#4{\arLaw{#1}{#2}{#3}{#4}\leq}
%------ using pstricks (rnode etc) ------------------------------------------
% \usepackage{pstricks,pst-node,pst-text,pst-3d}



\def\eqm{\mathbin{\equiv}}                     
\def\noeqm{\mathbin{\not\!\equiv}}  
%\newcommand{\flam}[2]{\lambda_{#1}\; .\; #2}
\def\existential#1#2{\exists_{#1}\;.\; #2}
\def\existencial#1#2{\exists_{#1}\;.\; #2}

\def\pv#1#2{\langle #1 \rangle #2}
\def\nc#1#2{[#1]#2}
\def\pvo#1#2{\langle \! \! \! \langle #1 \rangle \! \! \! \rangle\, #2}
\def\nco#1#2{\llbracket #1 \rrbracket #2}
\def\cvg#1{\llbracket \downarrow \rrbracket #1}
\def\cvgr#1#2{\llbracket #1 \downarrow \rrbracket #2}
\def\cvgl#1#2{\llbracket \downarrow  #1 \rrbracket #2}
\def\cvglr#1#2{\llbracket \downarrow  #1 \downarrow \rrbracket #2}
\def\lfp#1#2{\mu {#1}\, .\, {#2}}
\def\lpf#1#2{\mu {#1}\, .\, {#2}}
\def\gfp#1#2{\nu {#1}\, .\, {#2}}
\def\gpf#1#2{\nu {#1}\, .\, {#2}}
\def\mset#1{\vvv #1 \vvv}
\def\vvv{\vert \! \vert}
\def\mnc#1{\vvv [#1] \vvv}
\def\mpv#1{\vvv \langle #1 \rangle \vvv}
\def\bcomp#1{#1^{\text{c}}}
\def\eqm{\mathbin{\simeq}}
\def\noeqm{\mathbin{\not\!\simeq}}
\def\universal#1#2{\forall_{#1}\;.\; #2}
\def\existential#1#2{\exists_{#1}\;.\; #2}
\def\oexistential#1#2{\exists^{1}_{#1}\;.\; #2}
\def\MM{\mathcal{M}}
\def\uppaal{\textsc{Uppaal}}
\def\cc#1{\mathcal{C}(#1)}
\def\R{\mathcal{R}}
\def\TL#1{\mathcal{T}(#1)}
\def\ET#1{\mathsf{ExecTime(#1)}}

\def\pow#1{\ap{{\cal P}}{(#1)}}           \def\pow#1{{\cal P}#1} % P(X) 
\def\tran#1{\stackrel{#1}{\longrightarrow}}
\newcommand{\fdec}[3]{#1: #2 \longrightarrow  #3}


\begin{document}

\setLecture{4}{Verification of Real-Time systems in UPPAAL}
\frame[plain]{\titlepage}


\begin{slide}{Syllabus}
  \frsplitt{
    \begin{itemize}
      \item CSS: a simple language for concurrency
      \begin{itemize}
        \item Syntax
        \item Semantics
        \item Equivalence
      \end{itemize}
      \item  {Timed Automata}
      \begin{itemize}
        \item Syntax
        \item Semantics (composition, Zeno)
        \item Equivalence
        \alert{
        \item {UPPAAL tool}
        \begin{itemize}
           \item Specification
           \item CTL and Verification
         \end{itemize}
        }
      \end{itemize}
    \end{itemize}
  }{
    \begin{itemize}
      \item A simple C-like language
      \begin{itemize}
        \item Syntax
        \item Semantics (operational)
      \end{itemize}
      \item {Hybrid-language}: adding differential equations
      \begin{itemize}
        \item Syntax
        \item Semantics
        \item {Lince tool}
        \begin{itemize}
          \item Specification
          \item Analysis
        \end{itemize}
      \end{itemize}
      \item {Monads}: semantics with computational effects
        % \begin{itemize}
        %   \item Modelling Cyber-physical behaviour with Monads
        %   \item Hybrid Monad
        %   \item Monad Combination
        % \end{itemize}
    \end{itemize}
  }
\end{slide}


\begin{frame}{Table of contents}
  \setbeamertemplate{section in toc}[sections numbered]
  \tableofcontents[hideallsubsections]
\end{frame}




\section{Modelling in \uppaal}
%----------------------------------------------------------------------------------
\begin{slide}{\uppaal}
\small

... an \structure{editor},  \structure{simulator} and \structure{model-checker} for TA with \structure{extensions} ...

\structure{Editor.}
\vspace*{-3mm}
\begin{itemize}
\item \alert{Templates} and \alert{instantiations}
\item Global and local \alert{declarations}
\item \alert{System definition}
\end{itemize}

\structure{Simulator.}
\vspace*{-3mm}
\begin{itemize}
\item Viewers: \alert{automata animator} and \alert{message sequence chart}
\item Control (eg, \alert{trace} management)
\item Variable view: shows values of the integer variables and the clock constraints defining symbolic states
\end{itemize}

\structure{Verifier.}
\vspace*{-3mm}
\begin{itemize}
\item (\texttt{see next session)}
\end{itemize}

\end{slide}


%----------------------------------------------------------------------------------
\begin{slide}{Extensions (modelling view)}
\small

\begin{itemize}
\item \structure{templates} with \alert{parameters} and an \alert{instantiation mechanism}
\item \structure{data expressions} over \structure{bounded integer variables} (eg, \texttt{int[2..45] x}) allowed in \alert{guards},
\alert{assigments} and \alert{invariants}
\item rich set of \structure{operators} over integer and booleans, including bitwise operations, arrays, initializers ... in general
a whole \structure{subset of C} is available
\item \structure{non-standard} types of \alert{synchronization}
 \item \structure{non-standard} types of \alert{locations}
\end{itemize}
\end{slide}




%----------------------------------------------------------------------------------
\begin{slide}{Extension: broadcast synchronization}
\small



\begin{itemize}
\item A sender  can synchronize with an arbitrary number of receivers 
\item Any receiver than can synchronize in the current state must do so
\item Broadcast sending is never blocking (the send action can occur even with no receivers).
\end{itemize}

\end{slide}

%----------------------------------------------------------------------------------
\begin{slide}{Extension: urgent synchronization}
\small

\begin{columns}
  \column{0.3\textwidth}
     \includegraphics[width=\textwidth]{./images/urgent.jpg} 

  \column{0.7\textwidth}
    Channel $a$ is declared \structure{\texttt{urgent chan a}} if both edges are to be taken as soon as they are ready 
    (\alert{simultaneously} in locations $\ell_1$ and $s_1$).

    Note the problem can  \alert{not} be solved with \alert{invariants} because locations $\ell_1$ and $s_1$ 
    can be reached at different moments

    \begin{itemize}
    %\item Declare as \texttt{urgent chan a}
    \item No delay allowed if a synchronization transition on an urgent channel is enabled
    \item Edges using urgent channels for synchronization cannot have time constraints (ie, clock guards)
    \end{itemize}
\end{columns}
\end{slide}


%----------------------------------------------------------------------------------
\begin{slide}{Extension: urgent location}
\small

\begin{columns}
  \column{0.3\textwidth}
     \includegraphics[width=\textwidth]{./images/urgent2.jpg} 

  \column{0.7\textwidth}
    \begin{itemize}
    %\item Declare as \texttt{urgent chan a}
    \item Time does not progress but interleaving with normal location is allowed
    \item Both models are equivalent: \alert{no delay at an urgent location}
    \item but the use of \structure{urgent location} reduces the number of clocks in a model and simplifies analysis
    \end{itemize}
\end{columns}

\end{slide}


%----------------------------------------------------------------------------------
\begin{slide}{Extension: committed location}
\small\centering

% \begin{itemize}
% \end{itemize} 

\begin{columns}
\column{0.3\textwidth}
   \includegraphics[width=\textwidth]{./images/urgent1.jpg} 

\column{0.7\textwidth}
\begin{itemize}
\item delay is not allowed and the committed transition must be left in the next instant (or one of them if there are several), i.e., 
next transition must involve an outgoing edge of at least one of the committed locations
\item Our aim is to pass the value $k$ to variable $j$ (via global variable $t$)
\item Location $n$ is \structure{committed} to ensure that no other automata can assign $j$ before
the assignment $j:=t$
\end{itemize}
\end{columns}


\end{slide}

%%----------------------------------------------------------------------------------
%\begin{slide}{Modelling patterns: Value passing}
%\small
%
%\begin{center}
% \includegraphics[width=10cm]{pass.jpg} 
%\end{center}
%\end{slide}
%
%%----------------------------------------------------------------------------------
%\begin{slide}{Hints}
%\small
%
%
%\begin{itemize}
%\item \alert{Modelling patterns}: see the \uppaal\ tutorial
%\item \alert{Further examples}: see the \texttt{demo} folder in the standard distribution
%\end{itemize}
%
%\end{slide}
%
%%----------------------------------------------------------------------------------
%----------------------------------------------------------------------------------
\begin{slide}{The train-gate example}
\small
\begin{columns}

\begin{column}{5cm}
\begin{center}
% \includegraphics[width=5cm]{./images/tg1.jpg} 
\uppbox[5cm]{Train(id)}{./images/Train-ta.pdf}
\end{center}
\end{column}

\col[0.65]{
\begin{itemize}
\item Events model \alert{approach}/\alert{leave}, order to \alert{stop}/\alert{go}
\item A train cannot be stopped or restart instantly
\item After \alert{approaching}  \structure{it has 10m} to receive a \alert{stop}. 
\item After that it  takes further \structure{10m} to reach the cross
\item After \alert{restarting} takes \structure{7 to 15m} to reach the cross and \alert{3-5m} to \alert{cross}
\end{itemize}
}
\end{columns}
\end{slide}


%----------------------------------------------------------------------------------
\begin{slide}{The train-gate example}
\small

\frsplitdiff{0.48}{0.5}{
\begin{itemize}
\item Note the use of parameters and the select clause on transitions
\item Programming ...
\end{itemize}
}{
\begin{center}
\vspace*{-4mm}
% \includegraphics[width=5cm]{./images/tg2.jpg} 
\uppbox[6cm]{Gate}{./images/Gate}
\end{center}
}
\end{slide}

%\begin{frame}[fragile]{Fisher's mutual exclusion protocol}
%\small
%
%\begin{columns}
%\column{.4\textwidth}
%\includegraphics[width=1.1\textwidth]{./images/P.pdf} 
%
%\column{.58\textwidth}
%\begin{itemize}
%  \item Processes want to access a shared Critical Section
%  \item Send a request for the access
%  \item Wait $k$ time units -- if access is not given, resend a request
%\end{itemize}
%\end{columns}
%
%\bigskip
%
%\begin{lstlisting}[emph={[2]forall}]
%% P(1) requests access => it will eventually wait
%P(1).req -> P(1).wait
%% mutual exclusion invariant
%A[] forall (i:int[1,6]) forall (j:int[1,6])
%  P(i).cs && P(j).cs imply i == j  
%\end{lstlisting}
%
%\end{frame}


%%%%%%%%%%%%%%%%%%%%%%%%%%





\section{Behavioural Properties}
%----------------------------------------------------------------------------------
\begin{slide}{Properties: expression and satisfaction}
\small
\begin{block}{The satisfaction problem }
Given a \alert{timed automata}, $ta$, and a \structure{property}, $\phi$, show that
\begin{equation*}
\TL{ta} \models \phi
\end{equation*}
\end{block}
~\\

\pause
\begin{itemize}
\item in which logic language shall $\phi$ be specified?
\item how is $\models$ defined?
\end{itemize}
\end{slide}




%----------------------------------------------------------------------------------
\begin{slide}{Expressing properties: \textsc{Uppaal}}
\small

\begin{block}{\textsc{Uppaal} variant of \textsc{CTL}}
\begin{itemize}
\item \structure{state formulae}:  describes individual states in $\TL{ta}$
\item \structure{path formulae}: describes properties of paths in $\TL{ta}$
\end{itemize}
\end{block}

\end{slide}

%----------------------------------------------------------------------------------
\begin{slide}{Expressing properties: \textsc{Uppaal}}
\small

\begin{block}{State formulae}
\vspace*{-4mm}
\begin{align*}
% \Pi\; ::=\; & \Ac \Boxc\, \Psi\, \mid\, \Ac\Diamondc\, \Psi\, \mid\, \Ec \Boxc\, \Psi\, \mid\, \Ec \Diamondc\, \Psi\, \mid\, \Phi \leadsto  \Psi\
% % \tag{path}
% \\[2mm]
\Psi\; ::=\; & ta.\ell ~|~ g_c ~|~ {\transp{g_d}} ~|~ \texttt{deadlock} ~|~
  \texttt{not~}\Psi ~|~ \Psi \texttt{ or }\Psi ~|~ \Psi \texttt{ and }\Psi ~|~
  \Psi \texttt{ imply }\Psi
  % \tag{state}
\end{align*}

Any expression which can be evaluated to a boolean value for a state (typically involving the 
\alert{clock constraints} used for guards and invariants and similar constraints over integer variables):
\vspace*{-6mm}
\begin{center}
\texttt{x >= 8}, \texttt{i == 8 and x < 2}, ...
\end{center}
Additionally,
\begin{itemize}
\item \structure{$ta.\ell$} which tests \alert{current location}:  $(\ell, \eta) \models ta.\ell$ \\
provided $(\ell, \eta)$ is a state in $\TL{ta}$
\item \structure{$\texttt{deadlock}$}: $(\ell, \eta) \models \forall_{d \in \R_0^+} .\, \text{there is no transition from} \; \pair{\ell,\eta+d}$ 
\end{itemize}

\end{block}

\end{slide}



\begin{slide}{Exercises}
  \centering
  \uppbox[7cm]{Lamp}{./images/Lamp}

  \doExercise{Write a state formula}{
  \vspace*{-8mm}
  \begin{enumerate}
    \item The lamp is \texttt{low}
    \item Not \texttt{off} and $y>25$
    \item If it is \texttt{low} or \texttt{bright}, then $y\leq 3600$
  \end{enumerate}
  }
  
\end{slide}

%----------------------------------------------------------------------------------
\begin{slide}{Expressing properties: \textsc{Uppaal}}
\small

\newcommand{\Boxc}{\alert{\Box}}
\newcommand{\Diamondc}{\alert{\Diamond}}
\newcommand{\Ac}{\structure{A}}
\newcommand{\Ec}{\structure{E}}

\begin{block}{Path formulae}
\begin{align*}
\Pi\; ::=\; & \Ac \Boxc\, \Psi\, \mid\, \Ac\Diamondc\, \Psi\, \mid\, \Ec \Boxc\, \Psi\, \mid\, \Ec \Diamondc\, \Psi\, \mid\, \Phi \leadsto  \Psi\
% \tag{path}
% \\[2mm]
% \Psi\; ::=\; & ta.\ell ~|~ g_c ~|~ {\transp{g_d}} ~|~ \texttt{deadlock} ~|~
%   \texttt{not~}\Psi ~|~ \Psi \texttt{ or }\Psi ~|~ \Psi \texttt{ and }\Psi ~|~
%   \Psi \texttt{ imply }\Psi
%   \tag{state}
\end{align*}

where
\begin{itemize}
\item \structure{$A$, $E$} quantify (universally and existentially, resp.) over \structure{paths}
\item \alert{$\Box$, $\Diamond$} quantify (universally and existentially, resp.) over \alert{states in a path}
\end{itemize}
also notice that

\vspace*{-3mm}
\begin{align*}
 \Phi \leadsto  \Psi\; \abv\; \Ac \Boxc\, (\Phi \imp \Ac \Diamondc\, \Psi)
\end{align*}
\end{block}

\end{slide}




% \uppbox[2.8cm]{\Large $A \Box\, \varphi$}{./images/AA.jpg}~~
% \uppbox[2.8cm]{$A \Diamond \, \varphi$}{./images/AE.jpg}


\begin{slide}{Expressing properties: \textsc{Uppaal}}
\small \centering

\uppbox[2.8cm]{\Large $A \Box\, \varphi$}{./images/AA.jpg}~
\uppbox[2.8cm]{\Large $A \Diamond \, \varphi$}{./images/AE.jpg}~~~~~
\uppbox[2.8cm]{\Large $E \Box\, \varphi$}{./images/EA.jpg}~
\uppbox[2.8cm]{\Large $E \Diamond\, \varphi$}{./images/EE.jpg}

\end{slide}

%----------------------------------------------------------------------------------
% \begin{slide}{Expressing properties: \textsc{Uppaal}}
% \small \centering

% ~\\
% \begin{tabular}{cc}
%   \Large $A \Box\, \varphi$ & \Large $A \Diamond \, \varphi$ \\
%  \includegraphics[width=2.8cm]{./images/AA.jpg} &
%  \hspace{1cm} \includegraphics[width=2.8cm]{./images/AE.jpg}
% \end{tabular}

% ~\\[2mm]

% % \begin{block}{$E \Box\, \varphi$ and $E \Diamond\, \varphi$}
% \begin{tabular}{cc}
%   \Large $E \Box\, \varphi$ & \Large $E \Diamond\, \varphi$ \\
%  \includegraphics[width=2.8cm]{./images/EA.jpg} &   \hspace{1cm} \includegraphics[width=2.8cm]{./images/EE.jpg}
% \end{tabular}
% % \end{block}
% \end{slide}

%----------------------------------------------------------------------------------
\begin{slide}{Expressing properties: \textsc{Uppaal}}
\small \centering

\uppbox[5cm]{\Large $\varphi\, \leadsto\, \psi$}{./images/LeadsTo.jpg}

% \begin{block}{$\varphi\, \leadsto\, \psi$}
% \begin{center}
%   \Large $\varphi\, \leadsto\, \psi$ \\
%  \includegraphics[width=5cm]{./images/LeadsTo.jpg} 
%  \end{center}
% \end{block}

\begin{exampleblock}{Example}
  If a message is sent, it will eventually be received -- $\texttt{send(m)} \leadsto \texttt{received(m)}$
\end{exampleblock}

\end{slide}

%----------------------------------------------------------------------------------
\begin{slide}{Reachability properties}
\small

\begin{block}{$E \Diamond\, \phi$}
\structure{Is there a path starting at the initial state, such that a state formula $\phi$ is eventually satisfied?}

\begin{itemize}
\item  Often used to perform sanity checks  on a model:
\begin{itemize}
\item \alert{is it possible for a sender to send a message?}
 \item \alert{can a message possibly be received?}
 \item ...
 \end{itemize}
 \item  Do not by themselves guarantee the correctness of the protocol (i.e. \alert{that any message is eventually delivered}), 
but they validate the basic behavior of the model.
 \end{itemize}
\end{block}

\end{slide}


%----------------------------------------------------------------------------------
\begin{slide}{Safety properties}
\small

\begin{block}{$A \Box\, \phi$ and $E \Box\, \phi$}
\vspace{5mm}
\structure{Something bad will never happen}\\
 or \structure{something bad will possibly never happen}
\vspace{5mm}

Examples\\
\begin{itemize}
\item \alert{In a nuclear power plant the temperature of the core is always (invariantly) under a certain threshold}.
\item \alert{In a game a safe state is one in which we can still win, ie, will possibly not loose}.
\end{itemize}

In Uppaal these properties are formulated positively: something good is invariantly true.
\end{block}

\end{slide}

%----------------------------------------------------------------------------------
\begin{slide}{Liveness properties}
\small

\begin{block}{$A \Diamond\, \phi$ and $\phi\, \leadsto \, \psi$}
\vspace{5mm}
\structure{Something good will \emph{eventually happen}}\\
 or \structure{if \emph{something} happens, then \emph{something else} will eventually happen!}
\vspace{5mm}

Examples\\
\begin{itemize}
\item \alert{When pressing the on button, then eventually the television should turn on}.
\item \alert{In a communication protocol, any message that has been sent should eventually be received}.
\end{itemize}

\end{block}

\end{slide}



%\end{document}




\begin{slide}{Exercise: worker, hammer, nail - revisited}
\small
\begin{columns}
\col[.45]{
  \centering
  ~\\[2mm]
  % Requires \usepackage{graphicx}
  % ~~~~\includegraphics[width=45mm]{./images/WHN.jpg}
  \uppbox[30mm]{Worker}{./images/Worker.pdf}\\[2mm]
  \uppbox[30mm]{Hammer}{./images/Hammer.pdf}\\[2mm]
  \uppbox[30mm]{Nail}{./images/Nail.pdf}\\
}
\col[.54]{
  \doExercise{Write properties and explain them}{
    \vspace*{-4mm}
    \begin{enumerate}
      \item Using $E\Diamond$
      \item Using $E\Box$
      \item Using $A\Diamond$
      \item Using $A\Box$
      \item Using $\leadsto$
    \end{enumerate}
    (Practice in UPPAAL)
  }
}
\end{columns}
\end{slide}


\begin{slide}{Exercise: write formulas}
  \frsplitdiff{0.33}{0.66}{
  \centering
  \uppbox[4.8cm]{Lamp}{./images/Lamp}

  }{
  \vspace*{-2mm}
  \doExercise{Write formulas, and say which ones are true}{
    \vspace*{-8mm}
    % \begin{multicols}{2}
    \begin{enumerate}
      \item The lamp can become bright;
      \item The lamp will eventually become bright;
      \item The lamp can never be on for more than 3600s;
      \item It is possible to never turn on the lamp;
      \item Whenever the light is bright, the clock $y$ is non-zero;
      \item Whenever the light is bright, it will eventually become off.
    \end{enumerate}
    % \end{multicols}
    }
  }
\end{slide}



\section{Examples: proving mutual exclusion}
% \section{Case-study: proving mutual exclusion}



%----------------------------------------------------------------------------------
\begin{slide}{The train gate example (1/2)}
\small

\begin{columns}

\col[0.35]{
\begin{center}
% \includegraphics[width=4cm]{./images/tg1.jpg}
\uppbox[5cm]{Train(id)}{./images/Train-ta} 
\end{center}
}

\col[0.65]{
\begin{itemize}
\item \visible<2>{\texttt{E<> Train(0).Cross}}
\\ \structure{(Train 0 can reach the cross)}
\item \visible<2>{\texttt{E<> Train(0).Cross and Train(1).Stop}}
\\ \structure{(Train 0 can be crossing bridge while Train 1 is waiting to cross)}
\item \visible<2>{\texttt{E<> Train(0).Cross and}
\\ \texttt{~~~~~~~(forall (i:id-t)}
\\ \texttt{~~~~~~~~~~~~i != 0 imply Train(i).Stop)}}
\\ \structure{(Train 0 can cross bridge while the other trains are waiting to cross)}
\end{itemize}
}

\end{columns}
\end{slide}

%----------------------------------------------------------------------------------
\begin{slide}{The train gate example (2/2)}
\small

\begin{columns}

\col[0.35]{
\begin{center}
% \includegraphics[width=4cm]{./images/tg1.jpg}
\uppbox[5cm]{Train(id)}{./images/Train-ta} 
\end{center}
}

\col[0.65]{
\begin{itemize}
\item \visible<2>{\texttt{A[] Gate.list[N] == 0}}
\\ \structure{There can never be N elements in the queue}
\item \visible<2>{\texttt{A[] forall (i:id-t) forall (j:id-t) Train(i).Cross \&\& Train(j).Cross imply i == j}}
\\ \structure{There is never more than one train crossing the bridge}
\item \visible<2>{\texttt{Train(1).Appr --> Train(1).Cross}}
\\ \structure{Whenever a train approaches the bridge, it will eventually cross}
\item \visible<2>{\texttt{A[] not deadlock}}
\\ \structure{The system is deadlock-free}
\end{itemize}
}
\end{columns}

\end{slide}



%%----------------------------------------------------------------------------------
%\begin{slide}{The train gate example}
%\small
%
%\begin{center}
%\includegraphics[width=5cm]{./images/tg2.jpg} 
%\end{center}
%
%\begin{itemize}
%\item Note the use of parameters and the select clause on transitions
%\item Programming ...
%\end{itemize}
%
%\end{slide}





%%----------------------------------------------------------------------------------
%\begin{slide}{Demo}
%\small
%
%
%\begin{itemize}
%\item The \alert{train gate} case study (included in the \textsc{Uppaal} distribution).
%\end{itemize}
%
%\end{slide}
%
%


%----------------------------------------------------------------------------------
\begin{slide}{Mutual exclusion}
\small

\begin{block}{Properties}
\begin{itemize}
\item \structure{mutual exclusion}: \alert{no two processes are in their critical sections at the same time}
\item \structure{deadlock freedom}: \alert{if some process is trying to access its critical section, then 
eventually some process (not necessarily the same) will be in its critical section; similarly for exiting the critical section}
\end{itemize}
\end{block}
\end{slide}

%%----------------------------------------------------------------------------------
\begin{slide}{Mutual exclusion}
\small


\begin{block}{The Problem}
\begin{itemize}
\item 
Dijkstra's original asynchronous algorithm (1965) requires, for $n$ processes to be controlled,
$\mathcal{O}(n)$ read-write registers and $\mathcal{O}(n)$ operations.
\item
This result is a theoretical limit (proved by Lynch and Shavit in 1992) which compromises scalability.
\end{itemize}
\end{block}
\pause
\fbox{but it can be overcome by introducing specific \structure{timing constraints}}
%\end{block}

\begin{block}{Two \emph{timed} algorithms:}
\begin{itemize}
\item  \structure{Fisher's protocol} (included in the \textsc{Uppaal} distribution)
\item  \structure{Lamport's protocol}
\end{itemize}
\end{block}
\end{slide}


%----------------------------------------------------------------------------------
\begin{slide}{Fisher's algorithm}
\small

\begin{block}{The algorithm}
\begin{align*}
\mathsf{repeat} & \\
& \mathsf{repeat}  \\
&  \hspace{1cm} \mathsf{await} \; id = 0\\
& \hspace{1cm} id := i\\
& \hspace{1cm}  \mathsf{delay}(k)\\ 
& \mathsf{until}\; id=i  \\
& \text{\emph{(critical section)}}\\
& id := 0\\
\mathsf{forever} &
\end{align*}
\end{block}
\end{slide}


%%----------------------------------------------------------------------------------
%\begin{slide}{Alur \& Taubenfeld's algorithm}
%\small
%
%\begin{block}{The algorithm}
%\begin{align*}
%\mathsf{repeat} & \\
%& s:\;  x:=i\\
%&\mathsf{await}\, (y = 0)\\
%& y := i\\
%& \mathsf{if}\;  x \neq i \; \mathsf{then}\,\\
%&  \hspace{1cm} \mathsf{delay}(2d); \\
%&  \hspace{1cm}  \mathsf{if}\; y \neq i \; \mathsf{then}\,\mathsf{goto}\, s; \\
%&  \hspace{1cm}  \mathsf{await}\, (\neg z)\\
%&  \mathsf{else}\, z:= true;\\
%& \text{\emph{(critical section)}}\\
%& z := false\\
%& \mathsf{if}\;  y = i \; \mathsf{then}\; y:=0;\\
%\mathsf{forever} &
%\end{align*}
%\end{block}
%\end{slide}


%----------------------------------------------------------------------------------
\begin{slide}{Fisher's algorithm}
\small
\begin{block}{Comments}
\begin{itemize}
\item One shared read/write register (the variable $id$)
\item Behaviour depends crucially on the value for $k$ --- the \structure{time delay}
\item Constant $k$ should be \alert{larger than the longest time that a process may take to perform a step while trying to get access to its critical section} 
\item This choice guarantees that whenever process $i$ finds $id = i$ on testing the loop guard it can enter safely ist critical section: \alert{all} other processes are out of the loop or with their index in $id$ overwritten by $i$.
\end{itemize}
\end{block}
\end{slide}

%----------------------------------------------------------------------------------
\begin{slide}{Fisher's algorithm in \textsc{Uppaal}}
\small
\centering

% \includegraphics[width=6cm]{./images/P.pdf} 
\uppbox[5cm]{Fisher}{./images/P.pdf} 
%\includegraphics[width=1.1\textwidth]{./images/P.pdf} 


\begin{itemize}
\item Each process uses a local clock $x$ to guarantee that the upper bound between between its successive steps, while
trying to access the critical section, is $k$ (cf. \alert{invariant} in state $req$).
\item \alert{Invariant} in state $req$ establishes $k$ as such an upper bound
\item \alert{Guard} in transition from $wait$ to $cs$ ensures the correct delay before entering the critical section
\end{itemize}
\end{slide}

%----------------------------------------------------------------------------------
\begin{frame}[fragile]{Fisher's algorithm in \textsc{Uppaal}}
\small

\begin{block}{Properties}
%\includegraphics[width=12cm]{./images/ficher2.jpg} 
\begin{lstlisting}[emph={[2]forall,not}]
% P(1) requests access => it will eventually wait
P(1).req -> P(1).wait
% the algorithm is deadlock-free
A[] not deadlock
% mutual exclusion invariant
A[] forall (i:int[1,6]) forall (j:int[1,6])
   P(i).cs && P(j).cs imply i == j  
\end{lstlisting}
\end{block}

\begin{itemize}
\item The algorithm is \alert{deadlock-free}
\item It ensures  mutual exclusion if the correct timing constraints. 
\item ... but it is critically sensible to  small violations of such constraints: for example, replacing $x > k$ by 
$x \geq k$ in the transition leading to $cs$ compromises both \alert{mutual exclusion} and \alert{liveness}.
\end{itemize}
\end{frame}


%----------------------------------------------------------------------------------
\begin{slide}{Lamport's algorithm}
\small

\begin{block}{The algorithm}
\begin{align*}
\mathsf{start}: \; \;  &  a := i\\
& \mathsf{if}\; b \neq 0\; \mathsf{then\; goto\; start}\\
& b := i \\
& \mathsf{if}\; a \neq i\; \mathsf{then\; delay}(k)\\
& \hspace{12mm} \mathsf{else} \; \mathsf{if}\; b \neq i\; \mathsf{then\; goto\; start}\\
& \text{\emph{(critical section)}}\\
& b := 0
\end{align*}
\end{block}
\end{slide}

%----------------------------------------------------------------------------------
\begin{slide}{Lamport's algorithm}
\small
\begin{block}{Comments}
\begin{itemize}
\item Two shared read/write registers (variables $a$ and $b$)
\item Avoids \alert{forced waiting} when no other processes are requiring access to their critical sections
\end{itemize}
\end{block}
\end{slide}


%----------------------------------------------------------------------------------
\begin{slide}{Lamport's algorithm in \textsc{Uppaal}}
\small
\centering

% \begin{center}
% \includegraphics[width=9cm]{./images/lamport.jpg} 
\uppbox[80mm]{Lamport(pid)}{./images/lamport.jpg} 
% \end{center}

~\\[-6mm]

\hspace*{-130mm}%
\begin{tikzpicture}[remember picture, overlay]
  \node[draw=red,circle,very thick] at (242pt,125pt){~~~~};
  \node[draw=red,circle,very thick] at (242pt,69pt){~~~~};
  \node[draw=red,circle,very thick] at (159pt,51pt){~~~~};
  \node[draw=red,circle,very thick] at (128pt,76pt){~~~~};
  \node[draw=red,circle,very thick] at (132pt,107pt){~~~~};
\end{tikzpicture}

\end{slide}


%----------------------------------------------------------------------------------
\begin{slide}{Lamport's algorithm}
\small
\begin{columns}[T]

\col[0.43]{
\begin{block}{Model time constants:}
\begin{itemize}
\item \structure{$k$} --- time delay
\item \structure{$kvr$} --- max bound for register access
\item \structure{$kcs$} --- max bound for permanence in critical section 
\end{itemize}
Typically
\vspace{-10mm}
\begin{center}
\fbox{$k\; \geq \; kvr + kcs$}
\end{center}
\end{block}
}


\col[0.55]{
\begin{block}{Experiments}
\centering
\begin{tabular}{|l|c|c|c|c|}
\hline
& $k$ & $kvr$ & $kcs$ & verified? \\
\hline
Mutual Exclusion & 4 & 1 & 1 & Yes\\
Mutual Exclusion & 2 & 1 & 1 & Yes\\
Mutual Exclusion & 1 & 1 & 1 & No\\
No deadlock & 4 & 1 & 1 & Yes\\
No deadlock & 2 & 1 & 1 & Yes\\
No deadlock  & 1 & 1 & 1 & Yes\\
\hline
\end{tabular}
\end{block}
}
\end{columns}
\end{slide}





\end{document}
