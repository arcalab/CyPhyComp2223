\documentclass[11pt]{article}

%\renewcommand\familydefault{\sfdefault}
 
%input preamble and macros
\input{preamble}
\input{macros}

 
\usepackage[margin=1in]{geometry} 
\usepackage{amsmath,amsthm,amssymb}
\usepackage{hyperref}
\usepackage[linewidth=1pt]{mdframed}
%\newcommand{\N}{\mathbb{N}}
%\newcommand{\Z}{\mathbb{Z}}

% Named environments (no counters) 
\newenvironment{theorem}[2][Theorem]{\begin{trivlist}
\item[\hskip \labelsep {\bfseries #1}\hskip \labelsep {\bfseries #2.}]}{\end{trivlist}}
\newenvironment{lemma}[2][Lemma]{\begin{trivlist}
\item[\hskip \labelsep {\bfseries #1}\hskip \labelsep {\bfseries #2.}]}{\end{trivlist}}
%\newenvironment{exercise}[2][Exercise]{\begin{trivlist}
%\item[\hskip \labelsep {\bfseries #1}\hskip \labelsep {\bfseries #2.}]}{\end{trivlist}}
\newenvironment{problem}[2][Problem]{\begin{trivlist}
\item[\hskip \labelsep {\bfseries #1}\hskip \labelsep {\bfseries #2.}]}{\end{trivlist}}
\newenvironment{question}[2][Question]{\begin{trivlist}
\item[\hskip \labelsep {\bfseries #1}\hskip \labelsep {\bfseries #2.}]}{\end{trivlist}}
\newenvironment{corollary}[2][Corollary]{\begin{trivlist}
\item[\hskip \labelsep {\bfseries #1}\hskip \labelsep {\bfseries #2.}]}{\end{trivlist}}
 
% Environments with counters
\newtheoremstyle{myplain} {5mm}% ⟨Space above⟩
{3mm}% ⟨Space below⟩
{}% ⟨Body font⟩
{}% ⟨Indent amount⟩
{\bfseries}% ⟨Theorem head font⟩
{.}% ⟨Punctuation after theorem head⟩
{.5em}% ⟨Space after theorem head⟩2
{}% ⟨Theorem head spec (can be left empty, meaning ‘normal’)⟩

\date{}

\theoremstyle{myplain}
\newtheorem{exercise}{Exercise}

\theoremstyle{definition} % no italics
\newtheorem{subexercise}{}[exercise]

\newcommand{\subex}[1]{\begin{subexercise}#1\end{subexercise}}

\begin{document}
 
% --------------------------------------------------------------
%                         Start here
% --------------------------------------------------------------
 
\title{The Adventurers' Problem}
 
\maketitle

\noindent
In the middle of the night four adventurers encounter a shabby
rope-bridge spanning a deep ravine. For safety reasons, they decide
that no more than 2 people should cross the bridge at the same time
and that a flashlight needs to be carried by one of them in every
crossing.  They have only one flashlight. The 4 adventurers are not
equally skilled: crossing the bridge takes them 1, 2, 5, and 10
minutes, respectively. A pair of adventurers crosses the bridge in an
amount of time equal to that of the slowest of the two adventurers.
\begin{center}
\includegraphics[scale=0.8]{images/ropebridge-diag}
\end{center}
One of the adventurers claims that they cannot be all on the other
side in less than 19 minutes. One companion disagrees and claims that
it can be done in 17 minutes. Your task it to verify these claims in
\textsc{UPPAAL}\footnote{An animated description of the problem is
  available
  \href{https://www.youtube.com/watch?v=7yDmGnA8Hw0}{here}.}. Specifically,
your task is to,
\begin{enumerate}
\item \underline{model} the system above using what you learned about
  \underline{timed automata};
\item express in \underline{\textsc{CTL}} that it is
  \underline{possible} for all adventurers to be on the other side in
  17 minutes;
\item express in \underline{\textsc{CTL}} that it is
  \underline{impossible} for all adventurers to be on the other side
  in less than 17 minutes;
\item test these formulae in \textsc{UPPAAL}.
\end{enumerate}
\bigskip
\noindent
Some hints to help you get started: the fact that only one person can
carry the flashlight is a \underline{mutual exclusion problem}. 
Do \underline{not} try to model the whole system with just one timed
automaton. Instead use what you learned about parallel
composition. We suggest one timed automaton for the flashlight and
one timed automaton for each person.

\end{document}
