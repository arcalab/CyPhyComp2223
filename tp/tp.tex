\documentclass[11pt]{article}

%\renewcommand\familydefault{\sfdefault}
 
%input preamble and macros
\input{preamble}
\input{macros}

 
\usepackage[margin=1in]{geometry} 
\usepackage{amsmath,amsthm,amssymb}
\usepackage{hyperref}
\usepackage[linewidth=1pt]{mdframed}
\usepackage{proof}
%\newcommand{\N}{\mathbb{N}}
%\newcommand{\Z}{\mathbb{Z}}

% Named environments (no counters) 
\newenvironment{theorem}[2][Theorem]{\begin{trivlist}
\item[\hskip \labelsep {\bfseries #1}\hskip \labelsep {\bfseries #2.}]}{\end{trivlist}}
\newenvironment{lemma}[2][Lemma]{\begin{trivlist}
\item[\hskip \labelsep {\bfseries #1}\hskip \labelsep {\bfseries #2.}]}{\end{trivlist}}
%\newenvironment{exercise}[2][Exercise]{\begin{trivlist}
%\item[\hskip \labelsep {\bfseries #1}\hskip \labelsep {\bfseries #2.}]}{\end{trivlist}}
\newenvironment{problem}[2][Problem]{\begin{trivlist}
\item[\hskip \labelsep {\bfseries #1}\hskip \labelsep {\bfseries #2.}]}{\end{trivlist}}
\newenvironment{question}[2][Question]{\begin{trivlist}
\item[\hskip \labelsep {\bfseries #1}\hskip \labelsep {\bfseries #2.}]}{\end{trivlist}}
\newenvironment{corollary}[2][Corollary]{\begin{trivlist}
\item[\hskip \labelsep {\bfseries #1}\hskip \labelsep {\bfseries #2.}]}{\end{trivlist}}
 
% Environments with counters
\newtheoremstyle{myplain} {5mm}% ⟨Space above⟩
{3mm}% ⟨Space below⟩
{}% ⟨Body font⟩
{}% ⟨Indent amount⟩
{\bfseries}% ⟨Theorem head font⟩
{.}% ⟨Punctuation after theorem head⟩
{.5em}% ⟨Space after theorem head⟩2
{}% ⟨Theorem head spec (can be left empty, meaning ‘normal’)⟩

\date{}

\theoremstyle{myplain}
\newtheorem{exercise}{Exercise}

\theoremstyle{definition} % no italics
\newtheorem{subexercise}{}[exercise]

\newcommand{\subex}[1]{\begin{subexercise}#1\end{subexercise}}

\begin{document}
 
% --------------------------------------------------------------
%                         Start here
% --------------------------------------------------------------
 
\title{Cyber-Physical Computation
\\ {\small Last assignment}}
\author{José Proença and Renato Neves}
 
\maketitle

\subsubsection*{First task (managing shared resources with \textsc{Uppaal})}
Consider a small private airfield used by 2 planes, which can be either flying,
parked, landing, or taking off. The landing field is a shared resource by both
planes. Consider the following requirements:
\begin{itemize}
  \setlength\itemsep{0.3mm}
  \item only 1 plane can use the field at a time;
  \item a Controller component receives requests to \emph{land} or to \emph{take off}, and replies with a \emph{wait} signal when the field is not available;
  \item each plane sends requests to the Controller to \emph{land} or to \emph{take off}, and sends notifications when the field becomes \emph{free};
  \item the Controller has 5 time units to notify a plane to wait;
  \item after 5 time units from requesting access to the field and with no wait signal, the planes take another 5 time units to reach the field.
  \item each plane takes non-deterministically between 1-3 time units to take off, and between 4-6 time units to land and park.
  \item after taking off and after parking the planes notify the Controller with a \emph{gone} signal.
  \item if a plane is told to wait, we assume it will take between 5-7 time units to reach the field.
\end{itemize}

\noindent
Suggest a model for the planes and the controller. List 4 to 8 desired
properties that the model should satisfy. Verify the properties via
\textsc{Uppaal}.

\noindent {\underline{Extra points}:} Extend your model to handle $n$
planes at once.

\subsubsection*{Second task (small essay)}

Write a small essay detailing the differences between modelling and
verification (as you saw in \textsc{Uppaal}) and programming.  If possible
complement your explanations with a concrete running example.

\subsubsection*{Third task (program semantics and monads)}
Consider the following \emph{probabilistic} while-language.
\[
        \mathtt{Prog}(X) \ni \mathtt{x\ \blue{:=}\ t} \mid
        \mathtt{p} \, \blue{+}_p \, \mathtt{q} \mid
	\mathtt{p \> \blue{;} \> q} \mid
	\mathtt{\blue{if} \> b \> \blue{then} \> p \> \blue{else} \> q} \mid
	\mathtt{\blue{while} \> b \> \blue{do} \> \{ \> p \> \}}
\]
Notice that it contains a new language construct: namely $\mathtt{p} \>
\blue{+}_p \> \mathtt{q}$ which runs $\mathtt{p}$ with probability $p$ and
$\mathtt{q}$ with probability $1 - p$.  Develop a new semantics for the
extended language and implement it in \textsc{Haskell} using the probability
monad.

Here are some suggestions to help you get started: use the code developed in
previous lectures and also the library with the probability monad (available on
the website). Regarding the semantics, start with the following rule for
sequential composition and then try to derive the others.
\begin{flalign*}
        \infer[(\text{seq})]{
                \langle \mathtt{p\ \blue{;}\ q}, \sigma \rangle \Downarrow
                \sum_{i}^n p_i \cdot \mu_i 
        }{
                \langle \mathtt{p}, \sigma \rangle \Downarrow 
                \sum_i^n p_i \cdot \sigma_i 
                \qquad
                \forall i \leq n. \, 
                \langle \mathtt{q}, \sigma_i \rangle \Downarrow \mu_i
        }
\end{flalign*}

\noindent {\underline{Extra points}:} Extend the semantics to handle a
selection of your favorite effects, for example delays, log messages,
or exceptions.

\bigskip

\begin{mdframed}  
  \myparagraph{What to submit:} A single report in PDF for all tasks 
  \textbf{and} all the relevant source files. 
  Send by email
  (\underline{nevrenato@gmail.com}) a unique zip file
  ``\bash{cpc2223-N1_N2.zip}'', where \bash{N1} and \bash{N2} are your
  student numbers. The subject of the email should be ``\bash{cpc2223
    N1 N2}''.

\myparagraph{Deadline:} 26th June 2023 @ 23h59
\end{mdframed}

\end{document}
